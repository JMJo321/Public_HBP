\documentclass[12pt]{article}
\usepackage{amsmath, graphicx, url, afterpage,setspace,amsthm,amssymb}

% LOAD ENDFLOAT PACKAGE
%\usepackage[nolists,tablesfirst]{endfloat}

% LOAD CAPTION PACKAGE
\usepackage[centerlast,bf]{caption}

% LOAD SUBFIGURE PACKAGE
\usepackage[position=above]{subfig}

% LOAD ACCENTS PACKAGE
%\usepackage{accents}

% TABLE MULTI ROW PACKAGE
\usepackage{multirow}

% BOOKTABS
\usepackage{booktabs,array}


% FOR LANDSCAPE PAGES
\usepackage{lscape}
\usepackage{afterpage}

% TO REFERENCE FOOTNOTES
\usepackage{footmisc}


% MAIN BIBLIOGRAPHY
%\usepackage[longnamesfirst]{natbib}
\usepackage[round]{natbib}
\setcitestyle{aysep={}}
\bibliographystyle{ecta}
\newcommand\cites[1]{\citeauthor{#1}'s\ (\citeyear{#1})}
%\bibpunct{(}{)}{;}{a}{}{,}
%\bibliographystyle{chicago}

% DECLARE ARGMIN
\DeclareMathOperator*{\argmin}{arg\,min}

%%%%% Directory path of image files
\graphicspath{{../../Paper/Figures/}}

\makeatletter
\def\input@path{{../../Paper/Figures/}}
\makeatother

% DEFINE FIGURE INSERT, CAPTION, AND NOTE COMMANDS
\newlength{\figwidth}
\newcommand{\figinpt}[2]{
    \settowidth{\figwidth}{\includegraphics[#1]{#2}}
    \setcaptionwidth{\figwidth}
    \centering
    \includegraphics[#1]{#2}}

\newcommand{\figcapt}[2][\linewidth]{
    \setcaptionwidth{#1}
    \centering
    \caption{#2}}

\DeclareTextFontCommand{\fignotefont}{\normalfont\footnotesize}
\newcommand{\fignote}[2][\linewidth]{
    \begin{minipage}[]{#1}
        \vspace{12pt}
        \fignotefont{#2}
    \end{minipage}}

\DeclareTextFontCommand{\tabnotefont}{\normalfont\footnotesize}
\newcommand{\tabnote}[2][\linewidth]{
    \begin{minipage}[]{#1}
        \vspace{12pt}
        \tabnotefont{#2}
    \end{minipage}}

% ADJUST PAGE MARGINS
\oddsidemargin 0in
\evensidemargin 0in
\textwidth 6.5in
\textheight 9in
\topmargin -.325in

% DEFINE NEW THEOREM-LIKE ENVIRONMENTS
\newtheorem*{ntheorem}{Theorem}
\newtheorem{theorem}{Theorem}
\newtheorem{exercise}{Exercise}
\newtheorem{definition}{Definition}
\newtheorem{result}{Result}
\newtheorem*{nassumption}{Assumption}
\newtheorem{assumption}{Assumption}
\newtheorem{lemma}{Lemma}
\newtheorem*{nlemma}{Lemma}




\title{{\large \textbf{The Economics of Time-Limited Development Options: The Case of Oil and Gas Leases}}}
\author{Evan Herrnstadt \\ Ryan Kellogg \\  Eric Lewis\thanks{Herrnstadt: Congressional Budget Office; evan.herrnstadt@cbo.gov. Kellogg: Harris Public Policy, University of Chicago; and National Bureau of Economic Research; kelloggr@uchicago.edu. Lewis: Bush School of Government and Public Service, Texas A\&M University; ericlewis@tamu.edu. We thank William Patterson, Pengyu Ren, Grant Strickler, and especially Nadia Lucas for outstanding research assistance. We are grateful for the many thoughtful suggestions from conference and seminar participants at Aalto, AEA, AERE, Arizona, BYU, CBO, CEPR, Chicago, Columbia, DOJ, Energy Institute at Haas Energy Camp, FDIC, Federal Reserve Board, GAO, Harvard, Heinz-Tepper IO Conference, Indiana, Mannheim, Michigan, Naval Academy, Northwestern, NYU, Penn, Penn State, Rice Workshop on Economic and Environmental Effects of Oil and Gas, SITE, Texas, Texas A\&M, Southern Economic Association, Toulouse, University of Utah, Utah Winter Business Economics Conference, Washington University, and the Winter Marketing-Economics Summit. The analysis and conclusions expressed herein are solely those of the authors and do not represent the views of the U.S. Congressional Budget Office.}}


\date{\vspace{3ex}  \today \\  }

\begin{document}

\pagenumbering{roman}

\maketitle

\thispagestyle{empty}	


\begin{abstract}
Oil and gas leases between mineral owners and extraction firms ubiquitously include royalty and primary term clauses. The royalty denotes the share of revenue that is paid to the mineral owner, and the primary term specifies the date by which the firm must complete a well, lest it lose the lease. Using data from the Louisiana shale boom, we first show that wells' drilling timing is substantially bunched just before lease expiration, raising the question of why leases include development deadlines that distort drilling decisions. We then develop a contracting model in which mineral owners face firms with private information and have the ability to contract on both realized revenue and drilling timing. We show that primary terms can increase both the owner's expected revenue and total surplus because they counteract the delay incentives imposed by the royalty.
\end{abstract}


\newpage

\pagenumbering{arabic}

%\doublespace
%\onehalfspacing
\setstretch{1.4}

\section{Introduction \label{sec:Intro}}

Because the owners of subterranean oil and gas often lack the expertise or financial capital necessary to extract their resources, they typically write contracts with specialized extraction firms to act as their agents. In the United States, as well as several other countries, these contracts take the form of mineral leases that ubiquitously contain royalty and ``primary term'' clauses, the latter of which are use-it-or-lose-it requirements for the firm. This paper is aimed at understanding the effects of these clauses on firms' activities and on the expected revenue obtained by firms and mineral owners, with an emphasis on the recent boom in U.S. oil and gas production from shale resources. 

An oil and gas lease grants a firm an option, but not an obligation, to develop the mineral owner's property by drilling wells and extracting the hydrocarbons. Upon signing a lease, the firm pays the owner a flat fee, known as a ``bonus''. The primary term specifies a period of time (typically 3 to 10 years) that the firm has to drill at least one well and commence production. If it does so, the lease is then ``held by production'' and enters a secondary term that lasts until the firm ceases production. During the secondary term, the firm may also drill additional wells on the parcel to increase its overall production rate. On the other hand, if the firm does not commence production by the end of the primary term, the lease terminates, and the mineral owner is then free to sign a new contract with another firm or re-contract with the original firm. 

The royalty specified in the lease dictates the percentage of the lease's oil and gas revenue that the firm must pay to the mineral owner. These royalties are often significant, as the royalty rate typically lies between 12.5\% and 25\%. \citet{bib:brown} estimates that royalty payments associated with the six largest U.S. shale plays totaled \$39 billion in 2014.

The royalty and primary term clauses distort firms' incentives regarding when to drill wells and how much effort to invest in fracking and well completion. The incentive to drill at least one well before primary term expiration has received considerable attention within the industry, with numerous reports of firms drilling unprofitable wells for the sake of holding their lease acreage. For instance, the {\it San Antonio Express News} reported in 2012 that ``many companies . . . are drilling quickly simply to meet the terms of their contract and keep their leases---not because they want to drill gas wells now'' \citep{bib:hiller}. Although royalties are less prominent in the news, they also distort firm decisions because they are a tax on revenue only, thereby driving a wedge between the firm's profit and total surplus.

We begin our analysis by studying data from the Haynesville Shale in Louisiana, where the institutional structure and data availability are conducive for studying lease terms. We discuss relevant institutional features of the Haynesville in section \ref{sec:Inst}, discuss our data sources in section \ref{sec:Data}, and then show in section \ref{sec:bunching} that there is substantial bunching of drilling in the months just prior to lease expiration.\footnote{To be concise in the introduction, we ignore here the pooling of leases into pooling units. We are clear about the distinction between leases and units in the body of the paper below.} We further show that many leases---especially those in less productive areas of the Haynesville---are characterized by having only a single well that was drilled just before lease expiration, suggesting that drilling in these areas was primarily motivated by holding acreage for future wells rather than by immediate profits.

Because the bunching of drilling that we observe suggests that primary terms distort firms' drilling decisions, we next explore theoretically why mineral owners might include these contractual terms in their leases. Section \ref{sec:AnalyticModel} presents an analytic model of lease design that builds off of insights from \citet{bib:laffonttirole1986} and \citet{bib:board}. In our model the firm has a hidden signal of the productivity of the lease, which leads to information rents. The firm, if it signs a lease, chooses when to drill and can also exert hidden costly effort that determines how much oil and gas is produced. Effort and cost are not contractible, so the contract can only be contingent on production, revenues, and the date of drilling.

We show that the mineral owner can maximize its expected payoff by offering the firm a menu of contracts that include two types of contingent payments: a royalty and a drilling subsidy. The royalty, per intuition from \citet{bib:riley}, \citet{bib:hendricksporterguofo}, \citet{bib:bhattacharya}, and \citet{bib:ordin}, reduces the firm's information rent but also delays drilling and reduces the firm's hidden effort. Our contribution is to show that the owner can partially mitigate the royalty-induced delay by also including in the contract a drilling subsidy that is paid to the firm once it drills a well.

In practice, however, oil and gas leases almost always include primary terms rather than drilling subsidies. A primary term, like the drilling subsidy, counteracts the royalty by bringing drilling forward in time, but it does so in a coarser way that leads to the bunching that we observe in the data. We discuss a variety of reasons why primary terms rather than drilling subsidies may be used in practice, including owner liquidity constraints and preferences for simpler contracts.

We next consider whether primary terms are a viable second-best tool for mineral owners to increase their expected revenues, relative to a royalty-only infinite horizon lease. We address this question using a computable model that we present in section \ref{sec:CompModel}. The mineral owner makes a take-it-or-leave-it offer to the firm that includes a bonus payment, royalty, and possibly a primary term. As with the analytic model, the firm has private information on expected gas production and solves a drilling timing problem once the lease is signed. Our modeling of this problem builds off of prior work on the determinants of drilling timing, including \citet{bib:kellogg}, \citet{bib:bhattacharya}, \citet{bib:ordin}, and \citet{bib:agerton}. Two new features of our model are: (1) the inclusion of completion (fracking) effort choice, following \cites{bib:covert} fracking production function; and (2) the incorporation of the possibility that the firm and mineral owner may agree to extend the lease upon expiration of the primary term, subject to payment of a second bonus.

In section \ref{sec:Calibration}, we discuss how we calibrate our computational model to the median lease in our Haynesville data. Then, in section \ref{sec:Results}, we use the model to explore the economic effects of different lease terms, beginning with a simple case in which the lease can only accommodate a single well. We find that the optimal contract includes not just a royalty but also a finite primary term, and that including the primary term improves the mineral owner's expected revenues by      2.4\% ($\$      104 $\unskip ,000) relative to a royalty-only contract. In addition, we find that the optimal primary term improves not just the owner's value but also total (owner + firm) surplus. These improvements occur despite the discontinuous drop in drilling probability that the contract induces at the primary term's expiration date.

We also show that if the owner is able to offer a drilling subsidy, it can shift drilling forward without creating a bunching distortion. In line with our analytic model, a contract with an optimally-set drilling subsidy modestly improves the owner's expected revenue relative to the contract with an optimal primary term.

Finally, we enrich the model to examine the effects of primary terms when a single well can hold a lease (or collection of leases) upon which additional future wells may be drilled. We show that primary terms in this situation no longer convey a substantial revenue benefit to the mineral owner because they do not create an incentive to accelerate drilling for any well other than the first one, and because the incentive to accelerate the drilling of the first well is too large. This result may help explain why mineral owners in Louisiana have litigated over Louisiana's unitization policies and why mineral owners in other states are adopting lease clauses that prevent firms from holding large amounts of acreage with a single well.

Our paper is predated by an extensive literature examining oil and gas auctions and establishing a theoretical foundation for understanding asset sales with contingent payments. \citet{bib:porter} and \citet{bib:haile} summarize studies of federal offshore oil and gas auctions. These papers have focused on how the informational environment affects bonus bidding and drilling programs, largely taking the royalty and primary term as given. The literature on auctions with contingent payments was recently surveyed in \citet{bib:skrzypacz}. Of this large literature, our work is most closely related to papers that study the tradeoff between limiting buyers' information rents and minimizing ex-post moral hazard. This tradeoff was examined in static contexts in \citet{bib:laffonttirole1986} and \citet{bib:mcafee}, and was extended to the study of sales of options in \citet{bib:board} and \citet{bib:cong}, both of which use oil and gas auctions as motivation. \citet{bib:bhattacharya} and \citet{bib:ordin} are then perhaps closest to our work, as they quantify optimal oil and gas royalties in auctions of state-owned parcels in New Mexico, accounting for how royalties reduce firms' information rents but delay drilling by winning bidders.

Our paper is distinct in that it is the first, to the best of our knowledge, to study how primary terms impact economic outcomes in these settings. We show that primary terms can serve as a complement to royalty clauses by mitigating the moral hazard in drilling timing that would otherwise be induced. The upshot is that optimally-set primary terms can increase not only the owner's expected revenue but also the total expected surplus, despite creating what appears to be inefficient bunching of drilling ex-post. We relate primary terms to other possible contract designs, such as subsidization of drilling, and we also examine how lease terms affect well completion (fracking) decisions conditional on drilling, rather than just drilling timing alone.

The U.S. shale oil and gas industry is a prominent and data-rich environment in which to study sales of time-limited development options, but the underlying economics we study are likely to be relevant in other settings. For instance, master franchise contracts in retail settings typically specify royalty payments to the franchisor and impose a finite period of time for the franchisee to develop a minimum number of units \citep{bib:kalnins}. Licenses for adaptations of creative works (such as adaptations of novels for screenplays) often allow producers only a finite period to commence or complete production, lest the property rights revert to the original author \citep{bib:litwak}. And U.S. Federal Communication Commission spectrum auctions impose buildout requirements upon winning firms \citep{bib:cramton,bib:GAO}. Our hope is that this paper can serve as a springboard for studying the economics of contract term length in these and other settings.



\section{Institutional background \label{sec:Inst}}

\subsection{The Haynesville Shale}

Our study focuses on the Haynesville Shale, a shale gas formation in northwest Louisiana and east Texas. The development of new techniques combining horizontal drilling with hydraulic fracturing made it profitable to extract from the Haynesville, and speculation and drilling in the Haynesville surged in early 2008. The same technology led to drilling booms in other shale formations throughout the United States.

We focus on the Haynesville Shale---and specifically the Louisiana portion of the Haynesville---for two reasons. First, the Haynesville produces almost exclusively dry natural gas. The near-absence of liquids production allows us to focus our analysis on a single output. Second, the economic and legal institutions in Louisiana that shape leasing and the pooling of leases into units facilitate our empirical work, which requires us to match wells to their pooling units and associated leases. We summarize these institutions below.


\subsection{Background on leases and pooling units}

When a firm is interested in drilling on privately-owned land, it must negotiate a lease with the mineral owner.\footnote{The lease structure we describe here also applies nearly ubiquitously to publicly-owned oil and gas in the U.S. The primary difference between private and public leasing---as highlighted in \citet{bib:covertsweeney}---is that public leases are usually allocated by organized bonus bid auctions (rather than unstructured negotiations), in which the royalty and primary term are fixed in advance.} Since at least the early 20th century, U.S. oil and gas leases have regularly included a cash bonus paid at signing, a royalty, and a primary term \citep{bib:smith}. The royalty rate denotes the fraction of oil and gas revenue that must be paid to the mineral owner, and the primary term sets the amount of time that the firm has an option to drill and commence production before it loses the lease. If the firm drills a productive well before the lease expires, the lease is ``held by production'', which means that the firm continues to hold the lease as long as it maintains commercial oil and gas production. A lease may also include an extension clause, which gives the firm an option to extend the lease for a pre-specified amount of time in exchange for an additional, pre-specified payment to the mineral owner.

In practice, leases typically have a continuous operations clause that allows the firm to hold a lease beyond expiration, even if it is not producing, so long as it is actively in the process of drilling or ``completing'' a well (``completing'' means fracking the well and installing production equipment).\footnote{The level of activity sufficient to qualify as ``continuous operation'' can either be defined explicitly in the lease or be defined as common law good faith effort.} Our analysis will therefore focus on ``spudding'' a well---i.e., commencing drilling---as the necessary step to hold a lease beyond its primary term.

One problem for firms and regulators is that leases are often small relative to the area that is drained by a well, which in the shale era may have a horizontal length of 5,000 feet or more. Therefore, state regulators have established rules for combining leases into pooling units. In Louisiana, the default pooling unit for the Haynesville Shale formation is the square-mile section from the Public Land Survey System (PLSS). These units are created when leaseholding firms apply for and then receive a unitization order from the Louisiana Department of Natural Resources (DNR). Typically, multiple firms will hold leases within a given pooling unit, and drilling operations then effectively function as a joint venture. One lead firm, typically the one with the highest acreage share of leases, becomes the operating firm and decision maker. Costs and revenues are distributed to all leaseholding firms on an acreage-weighted basis. Each firm then has the obligation to distribute royalties on revenues to its mineral owners on an acreage-weighted basis.

Drilling a Haynesville well within a Haynesville pooling unit holds all current leases within the unit, not just those overlying the well itself. In addition, because horizontal wells in shale formations primarily recover only gas that is located in rock close to the well bore, square-mile units have space for multiple horizontal wells that run parallel to one another. Thus, drilling a single well in a unit grants the operating firm the indefinite right to drill additional wells within the same unit.

Owners of minerals that are unleased at the time of drilling---either because their parcels were never leased or because their leases expired prior to drilling---effectively become participants in the joint venture with acreage-weighted shares in the profits. Thus, firms lose the ability to earn profits from acreage that is unleased.\footnote{Because mineral owners typically do not have the financial liquidity to pay their share of the drilling and completion costs, Louisiana statute (LA R.S. 30:10) provides them the option not to pay. In that case, they do not receive their share of revenues until the well's overall revenues cover its costs (i.e., the well ``pays out''). As a consequence, firms cannot earn strictly positive profits from unleased acreage, except to the extent that they can ``pad'' costs in their cost reports to the state and to unleased mineral interests.} It is therefore the threat that acreage in a unit will convert from leased to unleased that gives firms an incentive to drill prior to the expiration of primary terms. A unit will typically consist of many leases, not all of which expire at the same time. The drilling incentive provided by a given lease's pending expiration then depends on the acreage of that particular lease as well as the schedule of expiration dates for remaining leases.\footnote{For instance, if the schedule of lease expirations is such that only one small lease expires today, with all other acreage expiring two years from now, the expiration of the small lease today provides only a small incentive to drill immediately.}


\section{Data sample and summary statistics} \label{sec:Data}

To examine how primary terms affect drilling timing, we use data on natural gas prices, rig dayrates, wells, leases, and units. This section summarizes these data and how we gather, clean, and merge them. Additional detail is provided in appendix \ref{appx:data}.

\subsection{Price and rig dayrate data} \label{sec:pricedayratedata}

We obtain our measure of the price of natural gas from 12-month natural gas futures prices for delivery at Henry Hub, Louisiana.\footnote{We use prices for delivery at a 12-month horizon because wells produce gas gradually rather than instantaneously and because 12 months is the longest horizon at which futures are consistently liquidly traded.} For the 2009 to 2013 period (during which most of the Haynesville drilling happens), the average natural gas price is \$5.07per mmBtu (mmBtu = million British thermal units), with a minimum of \$3.39and a maximum of \$7.75(all prices are deflated to December 2014 dollars).\footnote{We deflate all gas price, rig dayrate, and drilling cost data using the Bureau of Labor Statistics' Consumer Price Index for all goods less energy, all urban consumers, and not seasonally adjusted. The CPI series ID is CUUR0000SA0LE.} 

We also obtain---from Enverus (formerly DrillingInfo), a private industry intelligence firm---data on rig dayrates, which are the cost of renting a drilling rig for one day. The average dayrate from 2009--2013 was \$16,841(December 2014 dollars), with a minimum of \$12,470and a maximum of \$18,721\unskip.


\subsection{Well data} \label{sec:welldata}

We obtain data on well drilling and completions from the Louisiana Department of Natural Resources (DNR). These data include permit dates, spud dates, completion dates, the volume of water used in hydraulic fracturing, whether the well targets the Haynesville formation, and drilling and completion costs.\footnote{These drilling and completion costs come from drilling and completion cost reports that unit operators file with the Louisiana DNR for the purpose of determining severance taxes. These costs are tax deductible.} We obtain well-level monthly production data from Enverus.


% TABLE OF WELL SUMMARY STATS
\begin{table}[!t]
\centering
\figcapt{Summary statistics for wells}
\begin{tabular}{l c c c  c c c}\hline\hline
\multicolumn{1}{c}{Variable} & Obs & Mean & Std. Dev.
  & P10 & P50 & P90  \\ \hline
Well spud year & 2685 & 2010.5 & 1.5  & 2009 & 2010 & 2013 \\
Well completion year & 2685 & 2011 & 1.6  & 2009 & 2011 & 2013 \\
Accounting well cost (millions, Dec 2014\symbol{36}) & 2495 & 10.4 & 2.4  & 7.8 & 10.1 & 13.3 \\
Water volume (millions of gallons) & 2401 & 6 & 2.8  & 3.5 & 5.5 & 8.9 \\
PV total production (millions mmBtu) & 2484 & 3.6 & 1.5  & 1.8 & 3.5 & 5.4 \\
\hline\end{tabular}

\fignote[0.95\textwidth]{Note: Table includes all Haynesville wells, as defined in section \ref{sec:welldata} and appendix \ref{appx:decline}. The number of observations varies across rows due to missing values.}
\label{tab:sum_stat_wells}
\end{table}

We focus our analysis on wells that targeted the Haynesville formation. Summary statistics for these wells are given in table \ref{tab:sum_stat_wells}. Most Haynesville wells were spudded and completed between 2009and 2013\unskip. Water used in hydraulic fracturing ranged from less than 3.5million gallons to more than 8.9million gallons (these data are generally only available for wells drilled from 2010 onward). Reported drilling and completion costs (adjusted to December 2014 dollars) range from less than \$7.8million to more than \$13.3million.

To estimate the cumulative lifetime production from each well, we fit a decline curve to Enverus's monthly well-level production data.\footnote{Following \citet{bib:aks2018}, and consistent with \citet{bib:newell}, we assume that wells' production decline rate is unaffected by natural gas price shocks.} Our decline model is based on \citet{bib:patzek}, which derives a production decline functional form for shale gas wells.\footnote{Appendix \ref{appx:decline} discusses the details of our production decline model.} We use the estimated parameters to predict well-level production over time (extrapolating beyond our observed data) and then to predict the present value of each well's total lifetime cumulative production.\footnote{We use an annual discount factor of    0.909\unskip, consistent with \citet{bib:kellogg}.} Production summary statistics are shown in the last row of table \ref{tab:sum_stat_wells}. We find that the median present value of cumulative production is 3.5million mmBtu, with 10th and 90th percentiles of 1.8million mmBtu and 5.4million mmBtu, respectively. 



\subsection{Lease data} \label{sec:leasedata}

% TABLE OF LEASE SUMMARY STATS
\begin{table}[!t]
\centering
\figcapt{Summary statistics for leases}
\begin{tabular}{l c c c  c c c}\hline\hline
\multicolumn{1}{c}{Variable} & Obs & Mean & Std. Dev.
  & P5 & P50 & P95  \\ \hline
Year lease starts & 35331 & 2008.5 & 1.7  & 2005 & 2008 & 2011 \\
Year lease ends & 35331 & 2011.6 & 1.8  & 2008 & 2012 & 2014 \\
Primary term length (months) & 35331 & 37.2 & 6.4  & 36 & 36 & 60 \\
Royalty rate & 27754 & 23.1 & 3.4  & 18.8 & 25 & 25 \\
Indicator: Has extension clause & 35252 & .8 & .4  & 0 & 1 & 1 \\
Extension length (months) & 27405 & 24.1 & 2.9  & 24 & 24 & 24 \\
Area in acres & 35134 & 42 & 154.5  & .2 & 4.7 & 163.3 \\
\hline\end{tabular}

\fignote[0.9\textwidth]{Note: Table includes leases in our analysis sample of Haynesville units, as defined in section \ref{sec:poolingunitdata}. Number of observations varies across rows due to missing values.}
\label{tab:sum_stat_leases}
\end{table}

From Enverus, we compile data on the universe of oil and gas leases in Louisiana that started between 2002 and 2015. These data include the start date of the lease, the primary term, any extension options, the royalty rate, the lease's PLSS section, and the acreage of the lease. The initial signing bonus is not recorded. 

We focus on leases that are within our sample of Haynesville pooling units, as described in section \ref{sec:poolingunitdata} below. Table~\ref{tab:sum_stat_leases} presents descriptive statistics for the 35,331leases in this sample. Leases typically started between 2005and 2011\unskip. Most leases have 36month primary terms, and royalties are typically between 19\unskip\% and 25\unskip\%, with 25\unskip\% the most common royalty rate. About 78\unskip\% of leases have extension clauses, with the vast majority of extensions lasting 2years. Exercising the extension option requires the payment of an additional bonus, but these bonuses are not usually recorded in the lease documents. Finally, leases range from less than 0.19acres to more than 160acres, with a mean of about 42acres.


\subsection{Pooling unit data} \label{sec:poolingunitdata}

We obtain shapefiles for designated Haynesville units from the Louisiana DNR. These units are typically PLSS square-mile (640 acre) sections, though some units have slight irregularities.\footnote{We drop a small number of units that have fewer than 580 acres or more than 680 acres.} In addition to these DNR-designated Haynesville units, we also include in our sample PLSS sections that that lie within the convex hull of the DNR-designated units. 

% UNIT SAMPLE MAP AND # OF WELLS PER UNIT
\begin{figure}[!t]
\figcapt{Map of Louisiana Haynesville units}
\mbox{\subfloat[Units in analysis sample]{\figinpt{width=.47\textwidth,clip}{haynesville_units_in_sample.pdf}}}
\mbox{\subfloat[Wells drilled per unit]{\figinpt{width=.47\textwidth,clip}{well_count_in_haynesville_units.pdf}}}
\fignote[\textwidth]{Note: Panel (a) is a map of Haynesville units (each square is a unit), where units that are in the analysis sample are colored dark. The red rectangle is the outline of the map in figure~\ref{fig:lateral_path_operators_sample_map}. Panel (b) is a map of Haynesville units, with units colored by how many Haynesville wells were drilled as of March 2017.}
\label{fig:map_sample_wells}
\end{figure}

Since we are interested in how the incentive to hold acreage affects the drilling of Haynesville wells, we remove from our sample units that may be held by drilling or production from other oil and gas formations.\footnote{We are especially concerned that wells drilled into the Cotton Valley formation, which sits above the Haynesville, may hold Haynesville acreage. Individual leases may or may not contain ``vertical Pugh clauses'' that restrict the ability of a lessee to hold acreage in one formation by drilling into another formation at a different depth. Because we do not observe information on Pugh clauses in our data, we conservatively drop all units that might potentially be held by drilling or production in other formations.} We do so by dropping units that have leases executed prior to 2004, non-zero oil or gas production in 2006, or non-Haynesville wells drilled after 2000. Our remaining sample, which we refer to as our {\it analysis sample} because we use it for the analysis in section \ref{sec:bunching} below, includes 1,226units, which we map in panel (a) of figure~\ref{fig:map_sample_wells}.

We match leases to units using the reported section in each lease document.\footnote{In cases where a lease spans multiple sections and acreage within each section is not reported, we assume acreage is divided equally between the spanned sections.} In some cases, the reported total acreage of all leases in a unit exceeds the actual unit acreage due to likely duplicates in the data. We use a clustering procedure---described in detail in appendix \ref{appx:lease_data}---to identify and downscale acreage for these likely duplicates.\footnote{Duplicates typically arise in cases of undivided mineral interests, which occur when a parcel is jointly owned by the descendants of an initial owner and a separate lease is filed for each descendant.}

To match wells to units, we use GIS techniques to identify which unit the majority of a well's horizontal leg passes through. Further details are in appendix \ref{appx:wellunitmatch}. Figure~\ref{fig:lateral_path_operators_sample_map} shows a March 2017 snapshot of well laterals and pooling units for a selected portion of the Haynesville, illustrating the mapping of wells to units.

% HAYNESVILLE LATERAL MAP
\begin{figure}[!t]
\centering
\figcapt[\textwidth]{An example of drilling patterns in the Haynesville Shale.}
\includegraphics[scale=.7,clip=true,trim=1cm 4cm 1cm 4cm]{haynesville_hlegs_operator_map.pdf}
\fignote[.85\textwidth]{Note: Map produced using data from the Louisiana DNR's SONRIS. Each square is a unit, white dots are wellheads, and black lines are the approximate horizontal well path. Units are colored by unit operator. Data are as of March 2017 and include all Haynesville units, not just those in the analysis sample. The red rectangle in panel (a) of figure~\ref{fig:map_sample_wells} plots where in the Haynesville unit map this example is located.}
\label{fig:lateral_path_operators_sample_map}
\end{figure}


Table~\ref{tab:sum_stat_section} shows summary statistics for our analysis sample of units. Units tend to have their first lease expire between 2008and 2011\unskip, with a median of 2009\unskip. A total of 712units (58\unskip\%) have Haynesville wells drilled, with the first spud typically between 2008and 2011\unskip. Of the units with drilling, 74\unskip\% have only one well drilled, 15\unskip\% have 2 wells drilled, 5\unskip\% have 3 wells drilled, and 7\unskip\% have 4 or more wells drilled. The most wells we observe in a single unit is 18\unskip. Panel (b) of figure~\ref{fig:map_sample_wells} maps the number of wells drilled per unit. 



% UNIT SUMMARY STATS
\begin{table}[!t]
\centering
\figcapt{Unit-level summary statistics}
\begin{tabular}{l c c c  c c c}\hline\hline
\multicolumn{1}{c}{Variable} & Obs & Mean & Std. Dev.
  & P5 & P50 & P95  \\ \hline
Section acres & 1226 & 641.7 & 13.7  & 620.2 & 642.7 & 662.3 \\
Year first lease starts & 1226 & 2006.5 & 1.4  & 2005 & 2006 & 2008 \\
Year first lease expires & 1226 & 2009.5 & 1.5  & 2008 & 2009 & 2011 \\
Number of Hay. wells & 712 & 1.6 & 1.6  & 1 & 1 & 4 \\
Year of first Hay. spud & 712 & 2009.8 & 1  & 2008 & 2010 & 2011 \\
\hline\end{tabular}

\fignote[0.8\textwidth]{Note: Table includes units in our analysis sample of Haynesville units, as defined in section \ref{sec:poolingunitdata}. Rows 4 and 5 only include units with non-zero drilling.}
\label{tab:sum_stat_section}
\end{table}


Finally, figure~\ref{fig:haynesville_pricecrash_hgas} presents time series aggregates, within our analysis sample of Haynesville units, for three major variables: the natural gas price, leasing, and drilling. This figure shows that the gas price and Haynesville leasing peaked in early 2008, but drilling did not peak until about two years later, shortly before many leases were to expire. This pattern suggests that primary terms may have had a significant effect on aggregate drilling activity in the Haynesville, a possibility we examine more directly in our bunching analysis below. 


% HAYNESVILLE TIME SERIES FIGURE
\begin{figure}[!t]
\centering
\vspace{10pt}
\figcapt[\textwidth]{Time series of the Henry Hub, Louisiana natural gas 12-month futures price, Haynesville leases signed, and the number of Haynesville wells spudded}
\figinpt{width=0.73\textwidth,clip}{haynesville_time_series_plot.pdf}
\fignote[0.9\textwidth]{Note: Panels showing leasing and well spudding only include activity in units in our analysis sample of Haynesville units, as defined in section \ref{sec:poolingunitdata}.}
\label{fig:haynesville_pricecrash_hgas}
\end{figure}


\section{Evidence on primary terms and bunching of drilling} \label{sec:bunching}

To study the role of lease expirations in motivating drilling in the Haynesville, we compare the date that the first Haynesville well is spudded in each unit in our analysis sample to the first date that a lease within the unit reaches the end of its primary term. Figure~\ref{fig:drillprob_first_spud_vs_first_lease_expire}, panel (a) presents a kernel-smoothed distribution of spud timing relative to that expiration date, along with a 95\% confidence interval. Panel (b) of figure~\ref{fig:drillprob_first_spud_vs_first_lease_expire} presents a histogram of the same data. The substantial spike in the density prior to the expiration of the first lease suggests that lease expiration is frequently a binding constraint. In appendix \ref{appx:bunching}, we use a formal bunching test to confirm that this spike is both large and statistically significant. Appendix \ref{appx:drlginputs} further shows that wells drilled just before expiration are fully-completed, producing wells: they do not exhibit remarkably low production, water inputs, or drilling costs.

Some leases in the Haynesville have a built-in extension clause that allows the firm to pay an additional bonus to extend the primary term by two years. Accordingly, both panels of figure~\ref{fig:drillprob_first_spud_vs_first_lease_expire} show a secondary spike in drilling two years after the primary term expires. Figure~\ref{fig:extension_vs_not} splits out our sample into units in which the first lease to expire had an extension clause versus those that did not. The figure shows that units with extensions had a less pronounced drilling spike prior to the expiration of the original primary term and a larger drilling spike prior to the expiration of the extension term two years later.


% SIDE-BY-SIDE KERNEL AND HISTOGRAM OF DRLG
\begin{figure}[!t]
\figcapt{Date of first drilling relative to first expiration date}
\mbox{\subfloat[Kernel density]{\figinpt{width=.47\textwidth,clip}{section_descript/drillprob_allsections_kdensity.pdf}}}
\mbox{\subfloat[Histogram]{\figinpt{width=.47\textwidth,clip}{section_descript/drill_allsections.pdf}}}
\fignote[\textwidth]{Note: Panel (a) is a kernel-smoothed estimate of the probability of drilling the first Haynesville well in a unit on a given date, relative to the expiration date of the first lease within the unit to expire. Panel (b) is a histogram showing the same data, in which each bar represents two months. Vertical lines are drawn at the date of first lease expiration and two years after first lease expiration.}
\label{fig:drillprob_first_spud_vs_first_lease_expire}
\end{figure}

Figures \ref{fig:drillprob_first_spud_vs_first_lease_expire} and \ref{fig:extension_vs_not} indicate that drilling sometimes occurs after the first lease in the unit expires. It may be rational for operators to drill after the first lease expires if that lease does not account for a large share of the overall leased acreage and the remaining leases do not expire for some time. We examine this possibility in appendix \ref{appx:expirationdist} and find that, in line with firms' incentives, bunching of drilling at the first lease expiration is more pronounced when little time remains before most of the remaining leased acreage expires.


% FIGURE SEPARATING NO EXTENSION VS EXTENSION
\begin{figure}[!t]
    \centering
    \vspace{10pt}
    \figcapt[\textwidth]{Comparison of units where the first expiring lease had a built-in two-year extension clause versus units where the first expiring lease did not have any extension clause}
    \figinpt{width=.65\textwidth,clip}{section_descript/drillprob_by_section_ext_kdensity.pdf}
    \fignote[.85\textwidth]{Note: Figure shows kernel-smoothed estimates of the probability of drilling the first Haynesville well in a unit on a given date, relative to the expiration date of the first lease within the unit to expire. Vertical lines are drawn at the date of first lease expiration and two years after first lease expiration.}
    \label{fig:extension_vs_not}
\end{figure}

If the primary term is pushing firms to drill a well to hold leased acreage when they otherwise would not drill, we would expect that many units would have only a single well for an extended period of time. Indeed, of the units in our sample that have drilling, 74\unskip\% of them have only one well. Figure \ref{fig:map_sample_wells}, panel (b) and figure \ref{fig:lateral_path_operators_sample_map} similarly show that a large fraction of drilled units only had one well, even as late as March, 2017. These facts suggest that drilling of these wells was primarily motivated by the possibility of drilling additional future wells, rather than the prospect of immediate profits. We further examine this hypothesis in figure~\ref{fig:productive_vs_unproductive_onewell_vs_more}. Panel (a) compares units in the highest tercile of expected discounted cumulative production with those in the lowest tercile of production.\footnote{We discuss how we estimate unit-level productivity in section \ref{sec:Calibration}.} Low-productivity units are less likely to be drilled early in their underlying leases' primary term and more likely to be drilled just before the first lease expires. Panel (b) of figure~\ref{fig:productive_vs_unproductive_onewell_vs_more} uses the number of wells ultimately drilled as a proxy for productivity and similarly finds that units in which only one well is ever drilled are more likely to have had that well drilled just prior to the first lease expiring.

% FIGURE SPLITTING HIGH PRODUCTIVITY VS LOW PRODUCTIVITY UNITS
\begin{figure}[!t]
\figcapt{Comparison of high productivity units to low productivity units}
\mbox{\subfloat[High vs low productivity units]{\figinpt{width=.47\textwidth,clip}{section_descript/drillprob_by_section_tercilesprod_kdensity.pdf}}}
\mbox{\subfloat[Multi-well vs single well units]{\figinpt{width=.47\textwidth,clip}{section_descript/drillprob_by_section_wellcount_kdensity.pdf}}}
\fignote[\textwidth]{Note: Both panels present kernel-smoothed estimates of the probability of drilling the first Haynesville well in a unit on a given date, relative to the expiration date of the first lease within the unit to expire. Vertical lines are drawn at the date of first lease expiration and two years after first lease expiration. Panel (a) compares units in the top tercile of expected gas production versus the bottom tercile. Our calculation of unit-level expected production is discussed in section~\ref{sec:Calibration}. Panel (b) compares units in which there were multiple wells drilled versus units with just one well drilled.}
\label{fig:productive_vs_unproductive_onewell_vs_more}
\end{figure}

The variation in shading in figure \ref{fig:lateral_path_operators_sample_map} represents different unit operators. We find that operators often have control of multiple contiguous units, which suggests that the drilling patterns are not being driven by externalities like common pool inefficiencies or information spillovers (as in \citet{bib:hendrickskovenock}, \citet{bib:hendricksporter}, \cite{bib:lin}, and \cite{bib:hodgson}). We examine this possibility further in appendix \ref{appx:infoexternality}, where we show that there is a spike in drilling prior to primary term expiration regardless of whether or not the unit operator controls nearby units.


\section{An analytic model of oil and gas lease design \label{sec:AnalyticModel}}

Section \ref{sec:bunching} presented evidence that primary terms in oil and gas leases distort firms' decisions, leading to bunching of drilling at lease expiration. This section presents a model that attempts to shed light on why mineral owners might write leases that include a primary term, in spite of the ex-post distortion that it may generate.

Our model is rooted in a principal-agent framework---with the mineral owner as the principal and the firm as the agent---in which the firm possesses private information about the expected productivity from drilling a well. This private information might reflect the firm's superior geologic knowledge of the subterranean formation itself, or information about the productivity of its own drilling and completion techniques.\footnote{\citet{bib:covert}, for instance, finds that different well completion decisions can lead to substantially different levels of production.} The central tension in the model lies between the owner's desire to minimize the firm's information rent and the desire to not overly distort the firm's incentives to efficiently extract the gas.

To simplify exposition and more easily distill intuition, we assume that the owner proposes contracts to a single firm rather than to a set of competing firms. This assumption is consistent with evidence that most owners only talk to a single firm during the leasing process.\footnote{A 2010 survey of mineral lessors in the Marcellus Shale in Pennsylvania found that only 21\% of them spoke with more than one company before signing a lease \citep{bib:pennst}.} Relaxing this assumption (short of going all the way to a perfectly competitive model) would not qualitatively change our conclusions regarding the structure of the optimal contract.\footnote{\citet{bib:laffonttirole1987} and \citet{bib:mcafee} extend the single-agent \citet{bib:laffonttirole1986} model to include competition, showing that the contract structure is unchanged (though the contract selected by the winning firm becomes ``flatter'' in expectation as competition increases). Similar results hold with increased competition in \citet{bib:board}.}

How should the owner write the contract, assuming that it has the power to set the contractual terms?\footnote{Throughout the paper, we refer to an individual owner as ``it'', acknowledging the fact that mineral owners can sometimes be firms or holding companies.}$^,$\footnote{As \citet{bib:demarzo} explains, this assumption is not innocuous. If the owner holds all of the bargaining power, its optimal contract will be a contingent contract of the type discussed in this section. If instead the firm holds all of the bargaining power, the equilibrium contract will involve a flat cash transfer to the owner with no contingent payments. The fact that private oil and gas leases are not characterized by simple flat payments suggests that owners have some bargaining power (e.g., via the bilateral negotiation process in which they can make counter offers). Mineral owners also have bargaining power because their outside option (in Louisiana) is to be a working interest owner in their unit, which entitles them to a contingent payment that takes the form of a call option.} We assume that the contract can be contingent on two ex-post outcomes that are driven by the firm's choices: (1) the date at which the well is drilled; and (2) the quantity of gas extracted. In practice, both outcomes are observable and verifiable because drilling creates an obvious surface disturbance (and requires government permits) and because gas production is metered and reported to regulatory authorities. In contrast, we assume that the other dimensions of the firm's ``effort''---such as the quantity and quality of its fracking inputs---and the overall drilling and completion cost are non-contractible.

Given these contractibility assumptions, our model combines features of \citet{bib:laffonttirole1986} and \citet{bib:board}. \citet{bib:laffonttirole1986} presents a static model of procurement contracts that are contingent on ex-post cost reports. \citet{bib:board} models a setting in which a principal sells a development option to an agent, where the execution date is contractible but ex-post profits and revenue are not. Our approach combines the ex-post cost contractibility (in our case, ex-post revenue contractibility) of \citet{bib:laffonttirole1986} with the dynamic option execution model in \citet{bib:board}.

\subsection{Analytic model setup \label{sec:Setup}}

The key objects in our model are as follows:

\begin{itemize}
\item Time is discrete and denoted by $t\in\{0,...,T\}$, where $T$ is possibly infinite. The lease contract is set at $t=0$, and then starting at $t=1$ the firm can decide whether to execute the option to drill and complete a well. The owner observes the period in which both drilling and completion occur. Only one well may be drilled on the lease.
\item $e\in\mathbb{R}^+$ denotes non-contractible completion ``effort''. For instance, $e$ may represent the volume of fracking fluid used or engineering expenditures on well design.
\item The cost of drilling and completing the well is given by $c_0 + c_1e$, where $c_0$ and $c_1$ are strictly positive scalars that are common knowledge.
\item $\theta$ denotes the productivity of the natural gas resource should the firm drill and complete a well. $\theta$ is known by the firm but not by the owner.\footnote{Allowing the firm to have incomplete information about the true value of the underground reserve will not affect the paper's results, so long as the firm remains better-informed than the owner.} $F(\theta)$ denotes the owner's rational belief about the distribution of $\theta$. $F(\theta)$ has support on $[0,\bar{\theta}]$.
\item $y = g(e)\theta(1+\varepsilon)$ denotes the volume of natural gas extracted if the firm drills and completes a well with effort $e$. $y$ is contractible, and for simplicity we assume that $y$ is completely realized in the same period that the well is drilled and completed.\footnote{$y$ can be thought of as the present discounted volume of production over the life of the well, and accordingly $P_t$ can be thought of as the weighted average natural gas price that is expected to prevail over the life of a well that is completed at $t$. We assume that a well's production is not affected by gas price innovations subsequent to completion, consistent with \citet{bib:aks2018} and \citet{bib:newell}.} The production function $g(e)$ is common knowledge and maps completion effort onto a recovery ratio, with the properties that $g:\mathbb{R}^+\to[0,1)$, $g(0)=0$, $g'>0$, $\lim_{e\to 0^+}g'(e)\to\infty$, and $g''<0$. Finally, $\varepsilon$ is a mean-zero disturbance that is unknown prior to drilling by both the owner and firm. $\varepsilon$ is orthogonal to $e$ and $\theta$, and its distribution function $\Lambda(\varepsilon)$ is common knowledge.
\item The gas price at time $t$ is denoted $P_t$ and is common knowledge. The gas price evolves stochastically via a process that is common knowledge and has the property that $P_t$ is bounded above. $P^t$ denotes the entire history of prices from time 0 through $t$.
\end{itemize}

Both the owner and firm are risk neutral, share a common per-period discount factor $\delta$, and seek to maximize the expected present value of their respective cash flows. At $t=0$, the owner can offer a menu of contracts to the firm; the firm must then choose one such contract or decline entirely (yielding a payoff of 0). The contracts can specify transfers that are contingent on observed extraction $y$, the drilling and completion date $t$, and the history of prices $P^t$. After the firm selects its contract, it then faces an optimal stopping problem regarding when to drill, and when it drills it must choose its completion effort $e$.


\subsection{The revenue-optimal mineral lease \label{sec:AnalyticModelDisc}}

This section discusses the optimal contract implied by our model. We relegate its derivation---which draws heavily from \citet{bib:laffonttirole1986} and \citet{bib:board}---to appendix \ref{appx:optcontract}.

The mineral owner maximizes the expected present value of its cash flows by offering the firm a menu of contracts that include an up-front payment due at signing and a contingent payment that is affine in gas revenue. Specifically, the optimal contingent payment $z$ that is paid from the firm to the owner at well completion time $\tau$ takes the form
\begin{equation}
z(\theta,P_\tau y) = -z_0(\theta) + z_1(\theta)P_\tau y, \label{eq:contpayment}
\end{equation}

\noindent where $z_0(\theta)$ and $z_1(\theta)$ are both positively-valued, decreasing functions of $\theta$ (see equation (\ref{eq:CPay3}) in appendix \ref{appx:optcontract}). Both $z_0(\theta)$ and $z_1(\theta)$ are zero for $\bar{\theta}$, reflecting the standard intuition that incentives for the highest type are not distorted.

The intuition for the royalty $z_1(\theta)$ on production revenue in equation (\ref{eq:contpayment}) flows from the linkage principle \citep{bib:milgromweber, bib:riley, bib:demarzo} and has been discussed in prior work on oil leasing \citep{bib:hendricksporterguofo,bib:bhattacharya,bib:ordin}: Because revenue is correlated with the firm's private information $\theta$, the royalty reduces the firm's information rent by compressing the distribution of payoffs across types. A 100\% royalty is not desirable, however, because such a confiscatory royalty would result in the drilling option never being exercised. The optimal royalty therefore strikes a balance between reducing the firm's information rent and minimizing distortions to the firm's effort.

In standard mechanism design problems of this type, such as \citet{bib:laffonttirole1986}, ``effort'' is one-dimensional and unobservable. In oil and gas drilling, however, it is useful to (loosely) think of ``effort'' as having two dimensions: the decision of when to drill and the decision of how much labor, capital, and material to invest in the well. The former is straightforward for the mineral owner to observe and contract on, while the latter is not. The essence of our result in equation (\ref{eq:contpayment}) is that, given that the mineral owner wants to tax natural gas revenue, it can mitigate the resulting distortions to contractible dimensions of the firm's effort by subsidizing them. Specifically, the owner can mitigate royalty-induced delays in drilling by paying a subsidy $z_0(\theta)$ to the firm when it drills. In the extreme, if all dimensions of effort were contractible, the optimal mechanism would call for the owner to pay the full cost of the well and then receive 100\% of the production revenue.

\subsection{Primary terms versus drilling subsidies}

Oil and gas leases in practice employ a primary term in conjunction with the royalty, rather than the subsidy and royalty combination derived above. Per intuition from \citet{bib:board}, the royalty and subsidy in the optimal mechanism act as a Pigouvian tax and subsidy, in that they align the firm's profit-maximization problem with the owner's objectives. A primary term is qualitatively similar to a drilling subsidy, in the sense that both provide the firm with an incentive to drill sooner than it otherwise would, given the royalty. If the future path of natural gas prices were certain at $t=0$, the two policies would be able to achieve the same outcome. But gas prices in reality are stochastic, and per the intuition given by \citet{bib:weitzman} and \citet{bib:kaplow}, a ``quantity policy'' such as a primary term will result in an expected utility loss relative to a ``price policy'' that is an optimal Pigouvian subsidy. In particular, the primary term may result in drilling occurring too soon if realized gas prices are lower than expected, and it may result in drilling occurring too late if realized gas prices are higher than expected.

So why use a primary term rather than the subsidy? Mineral owners' liquidity constraints are likely to be one practical obstacle to the subsidy. Indeed, one reason why mineral owners contract with oil and gas firms in the first place is to address their inability to finance resource extraction themselves.\footnote{Liquidity constraints would seem to be less of a concern in the case of federal oil and gas leases, but in that case there may be political and budgetary constraints to having the government fund oil and gas drilling, particularly when the decision-making is under the control of the firm.} Even after receiving the bonus payment, it may not be possible for the owner to guarantee the payment of a significant subsidy to the firm at a time of the firm's choosing.\footnote{This problem could be remedied by holding the part of the bonus that would be used for the subsidy in escrow until drilling occurs. But doing so would impose costs by tying up capital in a low-return account for potentially many years.} Alternatively, a contract involving delayed royalty payments could closely mimic the optimal royalty and subsidy combination without requiring cash outlays by the owner. Alberta, Canada, has recently implemented such a system, though we are not aware of such contracts in U.S. private oil and gas leasing.\footnote{In Alberta's ``modernized royalty framework'', the government computes a well-level cost allowance based on industry average costs and basic well characteristics. It then receives a lower royalty rate from each well drilled until that allowance is paid down. See \url{https://www.alberta.ca/albertas-royalty-framework.aspx} for details.} 

The widespread use of primary terms may also be related to the fact that they are ``notched’’ policies---in the spirit of \citet{bib:kleven}---in that they impose a discontinuous change in the return to drilling a well just before versus just after the expiration date. As \citet{bib:kleven} notes, notched policies are widely used in public regulation and private contracts, despite their suboptimality in standard economic models.\footnote{For instance, health insurance policies typically impose open enrollment deadlines every calendar year rather than a continuously varying penalty for delayed elections.} \citet{bib:kleven} speculates that one reason behind the ubiquity of notches may be that individuals ``find discrete categories simpler and more intuitive than a continuum'' (p. 461). In this spirit, one might imagine how a mineral owner might more easily process the effect of a discrete deadline on drilling behavior than the subtler continuous and probabilistic effect of a subsidy.

Whatever factors drive the observed use of primary terms in lieu of drilling subsidies, it remains to assess whether primary terms can increase mineral owners' expected payoff relative to a royalty-only lease. Because modeling the returns from a lease with a primary term is not analytically tractable, we turn next to a computational model of drilling decisions, firms' profits, and mineral owners' revenues. This model will let us compare how these economic outcomes are affected by alternative lease designs, including the use of primary terms, drilling subsidies, or neither. It will also allow us to incorporate into our analysis the ability of firms in the Haynesville to hold an entire unit---and therefore the option to drill multiple future wells---by completing a single well. Section \ref{sec:CompModel} presents the structure of our computational model, section \ref{sec:Calibration} discusses how we calibrate it, and then section \ref{sec:Results} presents simulated economic outcomes from counterfactual lease designs.


\section{Computational model \label{sec:CompModel}}

The goal of our computational model is to assess how different lease terms affect drilling activity and the value that accrues to the firm and the mineral owner. This section summarizes the structure of the model. Additional detail is provided in appendix \ref{appx:comp_model_details}.\footnote{To simplify the exposition, this section ignores severance taxes, income taxes, and operating costs. We discuss these taxes and costs in section \ref{sec:Calibration} and appendices \ref{appx:comp_model_details} and \ref{appx:calibration}.} 

We begin by considering a simple case, aligned with our analytic model, in which a single mineral owner offers a lease contract to a single firm, and the lease can accommodate at most one well. We abstract away from the problem of pooling leases into a unit by assuming that the owner's acreage alone is sufficient for the well. As with the analytic model, the extraction firm has a type $\theta$ drawn from a distribution $F(\theta)$. While the firm knows $\theta$, the mineral owner only knows $F(\theta)$. 

At the beginning of the game ($t=0$), the owner makes a take-it-or-leave-it lease offer to the firm. The contract includes a primary term $\bar{T} \in \{\mathbb{N}, \infty\} $ that specifies the period in which the lease expires, a royalty rate $k_1 \in [0, 1]$ that specifies the fraction of revenues to be paid to the owner, and a bonus $R_1 \in \mathbb{R}^+$ that is paid to the owner when the firm signs the lease. The contract may also include a subsidy for drilling $S_1 \in \mathbb{R}^+$. We denote the vector of lease characteristics $\chi_1 = \{ \bar{T}, k_1, R_1, S_1 \}$. If the firm accepts the lease offer, it pays the bonus $R_1$ in period $t=0$ and the game continues.

In each period $t=1$ to $t = \bar{T}$, the firm chooses to drill one well or not. The payoff to drilling is affected by the gas price $P_t$ and drilling cost $C_t$, which evolve according to a common-knowledge first-order Markov process. If the firm drills, it chooses the profit-maximizing amount of water $W^*$ with which to frac the well, and it extracts the expected net present total production of the well $Y^* = \theta \cdot W^{*\beta}$, where $\beta\in(0,1)$. It also pays drilling costs $C_t$, pays the cost of water $P_w W^*$, collects the subsidy $S_1$ (if applicable), and pays royalties to the owner. If the firm does not drill, the game continues to period $t+1$.

When the lease is about to expire at $t = \bar{T}$, the owner makes a take-it-or-leave-it offer of an extension with contract terms $\chi_2 = \{ k_2, R_2, S_2 \}$. For computational simplicity, we assume that the extension term is infinite. We also assume in our simulations that $k_2 = k_1$ and $S_2 = S_1$, which is similar to how ``built-in'' extension clauses work in our Haynesville lease data. Faced with this new contract, the firm chooses either to drill, abandon the lease, or pay the extension bonus $R_2$ to extend the lease. If the firm pays to extend, then for periods $t \geq \bar{T} + 1$ the firm solves an infinite-horizon optimal stopping problem in which it must decide when to drill.

Table \ref{tab:comp_model_summary} summarizes the timing of the model, the order of actions, and the flow payoffs to the firm and the owner. For simplicity, the payoffs in this table abstract away from severance taxes, income taxes, and operating costs.\footnote{For the full payoffs, see equations (\ref{eq:static_pi}) and (\ref{eq:static_pi_nopay}) in appendix \ref{appx:comp_model_details}.} For each action $a$ and for all periods $t\geq1$, we add an action-specific cost shock $\nu^a_t$ to the per-period payoff of the firm. We assume that these shocks $\nu^a_t$ are drawn from an i.i.d. type 1 extreme value distribution with scale parameter $\sigma_{\nu}$.\footnote{We shift the distributions of the $\nu^a_t$ so that they are mean zero, so that in a period in which the firm does not drill with 100\% certainty, it receives zero flow payoff in expectation.} These idiosyncratic cost shocks capture unexpected transitory drivers of drilling behavior (such as drilling rig availability) and have the effect of smoothing the model's predicted time path of drilling.\footnote{For all periods $t \notin \{0, \bar{T} \}$, the possible actions are to drill ($a=1$) or not drill ($a =0$). For $t = \bar{T}$, we use a nested setup: First the agent chooses whether to drill ($a=1$, with additive shock $\nu^1_t$) or not. Then conditional on not drilling, the firm chooses whether to extend or abandon. We assume that the firm receives the same additive shock $\nu^0_t$ both for abandoning the lease and for continuing the lease.}

\begin{table}[!htbp]
\centering
\caption{Game structure: Timing, actions, and flow payoffs}
\begin{tabular}{l l l l l}
\hline \hline
\multicolumn{4}{l}{Lease stage: $\ell = 0$} & Time period: $t = 0$ \\
\hline
{} & \multicolumn{4}{l}{(1) Nature draws prices and costs}\\
{} & \multicolumn{4}{l}{(2) Owner chooses initial lease terms $\chi_1$, including $\bar{T}$}\\
{} & \multicolumn{4}{l}{(3) Firm chooses to accept contract $\chi_1$ or not}\\
{} & {} & \multicolumn{3}{l}{Accepts contract:}\\
{\; \;} & {\; \; \;} & {\; \;} & Firm payoff: & $-R_1$ \\
{} & {} & {} & Owner payoff: & $R_1$ \\
{} & {} & \multicolumn{3}{l}{Rejects contract:}\\
{} & {} & {} & Firm payoff: & $0$ \\
{} & {} & {} & Owner payoff: & $0$ \\
\hline \hline
\multicolumn{4}{l}{Lease stage: $\ell = 1$} & Time periods: $1 \leq t \leq \bar{T}-1$ \\
\hline
{} & \multicolumn{4}{l}{(1) Nature draws prices and costs}\\
{} & \multicolumn{4}{l}{(2) Firm chooses to drill or wait}\\
{} & {} & \multicolumn{3}{l}{Drills:}\\
{} & {} & {} & Firm payoff: & $(1 - k_1) P_t \theta W^{*\beta} - C_t - P_w W^*+ S_1  + \nu^1_t$ \\
{} & {} & {} & Owner payoff: & $k_1 P_t \theta W^{*\beta} - S_1$ \\
{} & {} & \multicolumn{3}{l}{Waits:}\\
{} & {} & {} & Firm payoff: & $ \nu^0_t$ \\
{} & {} & {} & Owner payoff: & $0$ \\
\hline \hline
\multicolumn{4}{l}{Lease stage: $\ell = 1$} & Time period: $t = \bar{T}$\\
\hline
{} & \multicolumn{4}{l}{(1) Nature draws prices and costs}\\
{} & \multicolumn{4}{l}{(2) Owner chooses extension lease terms $\chi_2$}\\
{} & \multicolumn{4}{l}{(3) Firm chooses to drill, pay bonus to continue, or abandon}\\
{} & {} & \multicolumn{3}{l}{Drills:}\\
{} & {} & {} & Firm payoff: & $(1 - k_1) P_t \theta W^{*\beta} - C_t - P_w W^*+ S_1  + \nu^1_t$ \\
{} & {} & {} & Owner payoff: & $k_1 P_t \theta  W^{*\beta} - S_1$ \\
{} & {} & \multicolumn{3}{l}{Continues:}\\
{} & {} & {} & Firm payoff: & $-R_2  + \nu^0_t$ \\
{} & {} & {} & Owner payoff: & $R_2$ \\
{} & {} & \multicolumn{3}{l}{Abandons:}\\
{} & {} & {} & Firm payoff: & $ \nu^0_t$ \\
{} & {} & {} & Owner payoff: & $0$ \\
\hline \hline
\multicolumn{4}{l}{Lease stage: $\ell = 2$} & Time periods: $t \geq \bar{T}+1$\\
\hline
{} & \multicolumn{4}{l}{(1) Nature draws prices and costs}\\
{} & \multicolumn{4}{l}{(2) Firm chooses to drill or wait}\\
{} & {} & \multicolumn{3}{l}{Drills:}\\
{} & {} & {} & Firm payoff: & $(1 - k_2) \theta  W^{*\beta} - C_t - P_w W^*+ S_2  + \nu^1_t$ \\
{} & {} & {} & Owner payoff: & $k_2 \theta  Y^*(\theta) - S_2$ \\
{} & {} & \multicolumn{3}{l}{Waits:}\\
{} & {} & {} & Firm payoff: & $ \nu^0_t$ \\
{} & {} & {} & Owner payoff: & $0$ \\
\hline \hline
\end{tabular}
\fignote[0.95\textwidth]{Note: Payoffs shown here ignore taxes and operating costs, and assume there are no unleased mineral interests. For the full payoffs, see equations (\ref{eq:static_pi}) and (\ref{eq:static_pi_nopay}) in appendix \ref{appx:comp_model_details}.}
\label{tab:comp_model_summary}
\end{table}

We solve the game computationally via backward induction. During the extension period $t > \bar{T}$, the firm makes optimal drilling timing decisions under an infinite horizon given contract terms $\chi_2$. In period $t = \bar{T}$, the owner forecasts the firm's future drilling decisions and chooses the extension bonus $R_2$ that maximizes the owner's payoff.\footnote{We simplify this step of the model by assuming that the mineral owner sets $R_2$ as though it still faces the original distribution of firm types. In principle, the owner should realize that a high-type firm would likely have already drilled by the time $\bar{T}$ is reached, and a low-type firm would not have accepted the original contract $\chi_1$. However, allowing the mineral owner to update its beliefs in this manner would result in an intractably complicated state space and a much larger computational burden, since the type distribution at $\bar{T}$ is a function of the complete realized price and drilling cost path during the primary term.} During the primary term periods $t < \bar{T}$, the firm forecasts the extension contract terms $\chi_2$ (including the extension bonus $R_2$) that the owner will offer in $t = \bar{T}$ and incorporates those terms into the expected payoff of waiting. In the initial time period $t=0$ the owner anticipates future drilling decisions and sets contract terms $\chi_1$, including the bonus $R_1$, that maximize its profits. Because we assume no random shocks in the initial period, there will be a threshold type $\theta^*$ such that all types with $\theta \geq \theta^*$ accept the initial contract.

In appendix \ref{appx:additional_wells}, we discuss how we extend the model to the case of multiple wells per unit, where a total of $M$ wells can be drilled in the unit. In this extension, we allow the firm to drill an additional $M-1$ wells in either the same period the first well is drilled or any period thereafter, since the initial well holds the lease by production.




\section{Calibration of the computational model \label{sec:Calibration}}

This section discusses how we calibrate the computational model, including stochastic price processes, drilling costs, the production function, and the distribution of productivity $F(\theta)$. Our goal is to calibrate the model to a representative pooling unit in the Haynesville shale in the first quarter of 2010, when data on wells' production and water input become regularly available. A summary of calibrated parameters is near the end of this section in table \ref{tab:sum_calibration}, and appendix \ref{appx:calibration} provides additional detail on our calibration procedures.

\paragraph{Natural gas prices and rig dayrates:}

We calibrate the stochastic process for natural gas prices $P_t$ using our natural gas price data series, aggregated to the quarterly level. Following \citet{bib:kellogg}, we assume that $P_t$ follows a Markov process given by equation (\ref{eq:P_markov_process}):

\begin{equation}
\ln P_{t+1} = \ln P_t + \kappa^P_0 + \kappa^P_1 P_t + \sigma^P \eta^P_{t+1},
\label{eq:P_markov_process}
\end{equation}

\noindent where the drift parameters $\kappa^P_0$ and $\kappa^P_1$ allow for mean reversion. We assume that price volatility $\sigma^P$ is constant. We estimate $\kappa^P_0$ and $\kappa^P_1$ by regressing $\ln P_{t+1} - \ln P_t$ on $P_t$, using data from 1993 (when futures prices are first reliably liquid) through 2009. The parameter estimates are shown in table \ref{tab:sum_calibration} and imply that the long-run mean natural gas price is \$4.99
\unskip /mmBtu.

Similar to the stochastic process for gas prices, we assume that rig dayrates $D_t$ follow the Markov process given by equation (\ref{eq:D_markov_process}):

\begin{equation}
\ln D_{t+1} = \ln D_t + \kappa^D_0 + \kappa^D_1 D_t + \sigma^D \eta^D_{t+1}.
\label{eq:D_markov_process} \\
\end{equation}

We assume that $\kappa^D_0 = \kappa^P_0$ and $\kappa^D_1 = \kappa^P_1\bar{D_t}/\bar{P_t}$, so that dayrate mean reversion is proportional to that of natural gas prices.\footnote{$\bar{P_t}$ and $\bar{D_t}$ denote the average price and dayrate, respectively, over 1993--2009.} These parameters imply that the long-run mean dayrate is \$9597
per day. We assume that the shocks $\eta^D_{t+1}$ and $\eta^P_{t+1}$ are each drawn from an i.i.d. bivariate standard normal distribution, with a covariance matrix that we estimate using the residuals of equations (\ref{eq:P_markov_process}) and (\ref{eq:D_markov_process}).



\paragraph{Taxes and operating costs:}

Louisiana imposes a        4\%severance tax on Haynesville shale wells that becomes payable after either the well has been producing for two years or the well's drilling costs have been paid, whichever comes first \citep{bib:kaiser}. Following \citet{bib:gulen}, we simplify this rule by assuming a tax of        4\%on production revenue and allowing the firm to deduct drilling costs (subject to revenue exceeding costs). The severance tax applies to both the firm's production and the mineral owner's royalty share.

Following \citet{bib:gulen}, we use a combined state and federal marginal corporate tax rate in Louisiana of     40.2\%\unskip. We treat 50\% of drilling and completion expenditures as immediately expensable, while the remainder must be capitalized and depreciated over time using the double declining balance method \citep{bib:metcalf}.\footnote{The effective marginal tax rate on drilling and completion costs is then     36.8\%\unskip.} We assume that the marginal tax rate of     40.2\%also applies to the mineral owner's royalty income. By equating the mineral owner's and the firm's tax rate on revenues, we avoid building into the model any tax advantages for shifting income from the firm to the mineral owner (or vice-versa). Moreover, all bonus payments and drilling subsidies in our simulations can then be interpreted as post-tax payments.\footnote{In reality, the marginal tax rate faced by mineral owners is likely to be heterogeneous given underlying heterogeneity in income from other sources. We also do not model the fact that royalties are tax-advantaged relative to bonus payments, since the firm must capitalize bonuses but can expense royalties.}

Finally, we follow \citet{bib:gulen} by assuming that operating and gathering costs are $\$     0.60 $/mmBtuthroughout the life of the well.\footnote{\citet{bib:gulen} calculates operating and gathering costs of \$0.55--\$0.60/mmBtu for Haynesville wells in the early years of operation, increasing thereafter as fixed operating costs are spread over declining production.} We also assume that operating and gathering costs are fully expensable for income taxes but not severance taxes.


\paragraph{Production function:}

The production function for the cumulative lifetime gas output $Y_i$ from well $i$ drilled at location $\{lon_i,lat_i\}$ using $W_i$ gallons of water is:

\begin{equation}
	Y(\theta_{lon_i,lat_i},W_i) = \theta_{lon_i,lat_i} W_i^{\beta} \varepsilon_i \label{eq:prodfn}
\end{equation}

Using our well-level lifetime cumulative production estimates, discussed in section \ref{sec:welldata}, we estimate the coefficient $\beta$ and a spatially smoothed distribution of unit-level productivities $\theta$ by applying the difference estimator described in \citet{bib:robinson} to the log of equation (\ref{eq:prodfn}). This procedure, described in detail in appendix \ref{appx:calibration}, yields an estimate of $\beta$ of     0.19and a set of unit-level expected log productivity values that are optimized for out-of-sample prediction using leave-one-out cross-validation. Figure \ref{fig:maps_productivity} displays the inputs and outputs of this procedure. Panel (a) shows cumulative well production, averaged within each unit.\footnote{See section \ref{sec:welldata} and appendix \ref{appx:decline} for a discussion of cumulative well production.} Panel (b) shows predicted unit-level production after spatial smoothing, evaluated at the sample mean level of water use.

% HAYNESVILLE UNIT PRODUCTIVITY FIGURE
\begin{figure}[!t]
\figcapt{Haynesville unit-level productivity}
\mbox{\subfloat[Actual average production per well]{\figinpt{width=.47\textwidth,clip}{unit_mean_production.pdf}}}
\mbox{\subfloat[Smoothed predictions of production per well]{\figinpt{width=.47\textwidth,clip}{unit_productivity.pdf}}}
\fignote[\textwidth]{Note: Panel (a) is a map of Haynesville units showing the calculated present value of aggregate well production, averaged within each unit, using decline estimation procedures discussed in section \ref{sec:Data} and appendix \ref{appx:decline}. Panel (b) plots the spatially smoothed distribution of unit-level production, evaluated at the sample mean level of water use, as discussed in section \ref{sec:Calibration}.}
\label{fig:maps_productivity}
\end{figure}


Following \citet{bib:covert}, our identification assumption is that $\log(\varepsilon_i)$ is orthogonal to $\log(W_i)$, conditional on $\theta_{lon_i,lat_i}$.\footnote{Our estimate of $\beta$ of     0.19is similar to the summed estimates of the coefficients on water and sand in columns (3) and (4) of table 5 in \citet{bib:covert}: 0.2544 and 0.2791, respectively.} If instead there are factors affecting the marginal productivity of water that are observed by the firm but not by us, our estimate of $\beta$ will be biased upward. We therefore examine the sensitivity of our main simulation results in section \ref{sec:Results} to the use of a lower value for $\beta$.\footnote{We find that the owner's optimal royalty increases when $\beta$ is smaller, in line with the intuition that when there is less opportunity for moral hazard in water input choice, the royalty can be increased. Decreasing $\beta$ has little impact on the optimal primary term. See table \ref{tab:sensitivity} in appendix \ref{appx:calibration}.}



\paragraph{Mineral owner's belief about the productivity distribution $F(\theta)$:}

We assume that the mineral owner believes that $\theta$ is distributed lognormally with mean and standard deviation parameters (on log($\theta$)) $\mu_{\theta}$ and $\sigma_{\theta}$, and we estimate these parameters from the distribution of unit-level productivities estimated above. We estimate $\mu_{\theta} = $   11.79and $\sigma_{\theta} = $    0.43\unskip.\footnote{The estimate of $\mu_{\theta}$ implies that at the calibration sample average water use of      6.0million gallons, a well drilled in the median unit can be expected to produce \input{single_numbers_tex/calibration/MedianmmBtu.tex}million mmBtu of gas (in present value).} 

The estimate of $\sigma_{\theta}$ is large in the sense that it implies that the mineral owner's 95\% confidence interval for log($\theta$) encompasses an interval of $\pm$    0.84log points around $\mu_{\theta}$. In essence, our calculation assumes that the mineral owner knows the distribution of predicted unit-level productivity across the entire Haynesville play but does not know where the productivity of its own parcel falls within this distribution. This approach will over-estimate mineral owners' uncertainty over $\theta$ to the extent that they have at least a rough knowledge of whether their parcel lies on a relatively productive or unproductive part of the Haynesville. We therefore examine and discuss in section \ref{sec:Results} how the use of smaller values for $\sigma_\theta$ affects the optimal royalties and primary terms from our simulations.


\paragraph{Water price, drilling costs, and idiosyncratic cost shocks:}

The remaining parameters to calibrate are the water price $P_w$, the level of drilling costs $C_{it}$ and their sensitivity to the rig dayrate $D_t$, and the scale parameter $\sigma_\nu$ that governs the variance of the idiosyncratic cost shocks $\nu^a_t$. We calibrate these parameters by fitting unit-level simulations of the firm's drilling problem to data on actual drilling timing, water use, and costs. This procedure uses only the components of our model needed to simulate drilling conditional on the unit's productivity, the contractual terms of the leases in the units, and the timing with which leases expire. It does not use the model components that track the owner's value or solve for the owner's optimal lease.

To calibrate these parameters, we start with our analysis sample, defined in section \ref{sec:poolingunitdata}, that excludes units that are held by drilling or production from non-Haynesville formations. Within this sample, the set of units that are suitable for calibration is limited by the facts that we do not regularly observe water use until 1Q 2010, and that data for some variables are missing or measured with error. We therefore restrict our sample to the      891units that were actively leased by 1Q 2010 but in which drilling had not yet occurred by 1Q 2010, and we then further drop units for which royalty rates, leased acreage data, or our productivity estimates are either missing or likely to be imprecise (see appendix \ref{appx:calibration} for details). Our final {\it calibration sample} then contains      160units.\footnote{Units in the calibration sample may be less productive, on average, than a typical Haynesville unit due to our selection procedure. We therefore examine the robustness of our main simulation results to increasing all units' productivity by 33\% (by increasing $\mu_\theta$). Table \ref{tab:sensitivity} in appendix \ref{appx:calibration} shows that our optimal royalty and primary term are qualitatively unaffected by this change, or by decreasing productivity by 33\%.}

We assume that the price of water $P_w$ is time-invariant and estimate it using the component of the model that computes the firm's static drilling profits and optimal water use, conditional on drilling. The estimated $P_w$ is that which minimizes the mean squared distance between profits (conditional on drilling) under optimal water use and observed water use, across all wells in the calibration sample. Our static profit function, discussed in greater detail in appendices \ref{appx:comp_model_details} and \ref{appx:calibration}, accounts for each unit's royalties and taxes, productivity, and share of acreage that is leased, along with the gas price and rig dayrate prevailing at the time of drilling. There are       96drilled wells in the calibration sample, including wells beyond the first well drilled in each unit. We estimate $P_w=$ $\$     0.39 $/gallon\unskip. This value should be thought of as the marginal cost of not just the water itself but also the associated labor and capital (e.g. pumping equipment) necessary to conduct the fracturing job.

To estimate the sensitivity of drilling costs to the rig dayrate, we project each well's reported cost $C_{it}$, less its cost of water $P_w W_i$, onto the rig dayrate $D_t$ per equation (\ref{eq:cost_day_rate_relationship}):\footnote{Because our calibration sample is concentrated in just a few quarters of 2010 and 2011 (see panel (a) of figure \ref{fig:calibrationfit}) we estimate $\alpha_1$ using the full sample of Haynesville wells for which we have reported cost and water input data.} 
\begin{equation}
C_{it} - P_w W_i = \alpha_0 + \alpha_1 D_t.
\label{eq:cost_day_rate_relationship}
\end{equation}
\noindent We estimate $\alpha_1 = $       57\unskip\space days. This value is inflated relative to actual drilling and completion times because the rig dayrate is one of several service costs that vary over time. Our projection in equation (\ref{eq:cost_day_rate_relationship}) essentially treats variation in rig dayrates as an index for variation in overall drilling and completion service costs.

Our cost data may systematically over or under-report true drilling and completion costs (firms have an incentive to over-report, but might also be unable to credibly report some costs such as engineering effort), so while we view these data as useful for estimating the slope term $\alpha_1$, we are wary of relying on them and the projection in equation (\ref{eq:cost_day_rate_relationship}) for our estimate of the drilling cost intercept $\alpha_0$.

% CALIBRATION SUMMARY TABLE
\afterpage{\begin{landscape}
\begin{table}[htbp]
\centering
\caption{Summary of parameters for model calibration}
\begin{tabular} {l c c l } \midrule \midrule 
\multicolumn{1}{c}{\textbf{Parameter}} & \textbf{Notation} & \textbf{Value} & \multicolumn{1}{c}{\textbf{Source}} \\ 
\midrule 
\textbf{State transitions} & $\{P_t, D_t\}$ & & Henry Hub prices and rig dayrates \\ 
\hspace{4pt} Price drift constant & $\kappa^P_0$ &   0.0058 &  \\ 
\hspace{4pt} Price drift linear term & $\kappa^P_1$ &  -0.0022 &  \\ 
\hspace{4pt} Price volatility & $\sigma^P$ &    0.102 &  \\ 
\hspace{4pt} Dayrate drift constant & $\kappa^D_0$ &   0.0058 &  \\ 
\hspace{4pt} Dayrate drift linear term & $\kappa^D_1$ & $-1.05 \times 10^{-6}$  &  \\ 
\hspace{4pt} Dayrate volatility & $\sigma^D$ &    0.093 &  \\ 
\hspace{4pt} Price - dayrate correlation & $\rho$ &    0.373 &  \\ 
\midrule 
\textbf{Taxes and operating / gathering costs} &  & & \citet{bib:gulen} \\ 
\hspace{4pt} Severance taxes & $s$ & 4\% of revenues & \\ 
\hspace{4pt} Federal and state income taxes & $\tau$ &     40.2\% of income & \\ 
\hspace{4pt} Effective income tax on capital expenditure & $\tau_c$ &     36.8\% of income & \\ 
\hspace{4pt} Operating and gathering costs & $c$ & $\$     0.60 $ / mmBtu & \\ 
\midrule \midrule 
\textbf{Well productivity} & $\theta \sim F(\theta)$ & mmBtu & Analysis of production and water use data  \\ 
\hspace{4pt} Coefficient on water input & $\beta$ &     0.19 & \\ 
\hspace{4pt} Mean of $\ln(\theta)$ & $\mu_{\theta}$ &    11.79 & \\ 
\hspace{4pt} SD of $\ln(\theta)$ & $\sigma_{\theta}$ &     0.43 & \\ 
\midrule 
\textbf{Drilling and completion costs} &  & &  \\ 
\hspace{4pt} Water price & $P_w$ & $\$     0.39 $ / gallon & Fit profits at optimal vs actual water use \\ 
\hspace{4pt} Dayrate coefficient & $\alpha_1$ &        57  days & Project reported well cost data onto dayrates \\ 
\hspace{4pt} Intercept & $\alpha_0$ & $\$     26.0 $ million & Maximum likelihood fit to drilling data \\ 
\hspace{4pt} Cost shock scale parameter & $\sigma_{\nu}$ & $\$      7.0 $ million & Maximum likelihood fit to drilling data \\ 
\midrule 
\end{tabular}
\label{tab:sum_calibration}
\end{table}
\end{landscape}}

Instead, we use maximum likelihood to jointly estimate the drilling cost intercept $\alpha_0$ and the scale parameter $\sigma_{\nu}$ of the cost shocks $\nu^{a}_t$. For a given pair of parameters, the model predicts the probability that the first well in each unit would be drilled in a given quarter (or not at all). The estimator fits these probabilities to the drilling decisions observed in our calibration sample (in which       71of      160units are drilled). The cost intercept $\alpha_0$ governs the overall level of drilling (with higher costs reducing drilling), while the scale parameter $\sigma_{\nu}$ governs how drilling within and across units responds to gas price shocks, differences in unit-level productivity, and anticipated expiration of leased acreage (with higher values of $\sigma_{\nu}$ reducing the sensitivity of drilling to these factors). We provide additional detail on our log-likelihood calculation in appendix \ref{appx:calibration}.

We find that large cost shocks are necessary to rationalize observed drilling: our estimate of $\sigma_{\nu}$ is $\$      7.0 $million.\footnote{The value of $\sigma_{\nu}$ of $\$      7.0 $million is pre-tax. When we compute the model, we scale down $\sigma_{\nu}$ for income taxes, severance taxes, and unleased acreage, just as we do for the other components of drilling costs in equation (\ref{eq:static_pi}) in appendix \ref{appx:comp_model_details}.} This value reflects the fact that our model abstracts away from factors such as rig availability, learning, and financial frictions that also affect firms' drilling decisions \citep{bib:covert,bib:agerton,bib:fetter_etal,bib:hodgson,bib:steck,bib:gilje}. Prior work that models firms' dynamic drilling problem has also estimated large shocks in order to fit the data \citep{bib:kellogg,bib:agerton}. 

The large estimate of $\sigma_{\nu}$ then drives a large estimate of the drilling cost intercept $\alpha_0$: $\$     26.0 $million. Because drilling in our model is often associated with the realization of a large negative cost shock, a large mean drilling cost intercept $\alpha_0$ is necessary to generate a reasonable mean drilling cost (excluding water costs) conditional on drilling: $\$      7.3 $million.\footnote{In the calibration sample, the average quarterly drilling hazard is      3.0\%\unskip. The expected value of $\alpha_0$ plus the cost shock $\nu^{a}_t$ for a well with a      3.0\%chance of being drilled is $\$      7.3 $million, which is similar to the intercept of $\$      7.2 $\unskip\space million estimated directly from projecting the cost data on dayrates per equation (\ref{eq:cost_day_rate_relationship}).}


Figure \ref{fig:calibrationfit} shows that the drilling simulated by the calibrated model fits the data reasonably well in both the time series and cross-section, though overall simulated drilling is slightly higher than actual drilling (      77units simulated versus       71units actually drilled). Table \ref{tab:sensitivity} in appendix \ref{appx:calibration} assesses the sensitivity of our simulation results to smaller values of $\sigma_{\nu}$ and $\alpha_0$, finding that the model's computed optimal royalty and primary term are qualitatively unaffected by decreases in these parameters.

% CALIBRATION FIT FIGURES
\begin{figure}[!t]
\begin{center}
\figcapt[\textwidth]{Fit of calibrated model to the drilling data}
\mbox{\subfloat[Time series of units drilled per quarter]{\figinpt{width=.47\textwidth,clip}{estimation/SimActDrillingVsTime.pdf}}}
\mbox{\subfloat[Cross section of drilling vs productivity]{\figinpt{width=.47\textwidth,clip}{estimation/SimActDrillingVsUnitProd.pdf}}}
\fignote[\textwidth]{Note: Panel (a) plots the number of times each quarter in which a unit is drilled for the first time, in both the actual data (circles) and calibrated simulation (solid line). Panel (b) plots the following unit-level variables against the unit's productivity coefficient $\theta$: whether the unit was ever actually drilled (circles plotted as 0 or 1), a lowess fit to actual drilling (solid line), the simulated probability the unit is ever drilled (x's), and a lowess fit to the simulated probability (dashed line). Plotted data include all units in the calibration sample.}
\label{fig:calibrationfit}
\end{center}
\end{figure}





\section{Results from simulation of alternative lease terms \label{sec:Results}}

The computational model presented in section \ref{sec:CompModel} and calibrated per section \ref{sec:Calibration} provides us with a tool to evaluate how drilling outcomes, the firm's profits, and the mineral owner's profits vary as a function of the lease's contractual terms. This section focuses on how the royalty and primary term shape drilling activity and the two parties' payoffs, with an emphasis on whether a finite primary term can improve both the mineral owner's surplus and total surplus. We also study an alternative lease design---in which the owner pays a drilling subsidy to the firm---that is more closely aligned with our analytical results in section \ref{sec:AnalyticModel}.


\subsection{Optimal contracts for simple units with one well \label{sec:onewellopt}}

We begin by studying the mineral owner's optimal contract for a simple lease, calibrated per table \ref{tab:sum_calibration}, on which only a single well can be drilled. For each set of lease terms we examine, we assume that the owner offers the firm a take-it-or-leave-it bonus payment that maximizes the owner's expected revenue conditional on the lease terms and the state variables (the gas price and rig dayrate) at the time of lease signing.\footnote{Recall that the owner knows only that the productivity of the lease is distributed $F(\theta)$, so the owner's revenues are an expectation taken over that distribution and over the future evolution of gas prices and rig dayrates.} We use a lease signing date of 1Q 2010, aligned with our calibration, at which time the gas price was $\$     6.70 $\unskip /mmBtu, and the rig dayrate was $\$    13064 $per day.

We find that the mineral owner's expected discounted revenue is maximized with a lease that has a       53\%royalty and a     4.75year primary term.\footnote{We find the optimal royalty and primary term via a nested search. Given a royalty, we find the discrete (and possibly infinite) primary term length that maximizes the owner's payoff. We then search for the optimal (continuous) royalty, optimizing the primary term at each iteration.} This contract improves the owner's expected revenue by $\$      104 $\unskip ,000 (     2.4\%\unskip) relative to a lease with a       53\%royalty and an infinite primary term (which generates $\$     4.32 $million in expected value for the owner).\footnote{The majority of the owner's value comes from the royalty rather than the bonus. The expected bonus revenue is $\$     1.96 $million with no primary term and $\$     1.03 $million with the optimal     4.75year primary term. On a per-acre basis (assuming a well drains one-third of a standard 640-acre unit), these bonuses are on par with anecdotally reported Haynesville bonus amounts of thousands or tens of thousands of dollars per acre \citep{bib:bogan}.} Figure \ref{fig:owner_priterm_1well}, panel (a) shows how the owner's expected value varies with the primary term, holding the royalty fixed at       53\%\unskip. The figure shows expected values not just for our central case in which the gas price at lease signing is $\$     6.70 $\unskip /mmBtu, but also cases in which the gas price is 33\% higher or lower. We normalize the owner's value in each case by presenting it as a share of the total surplus that would be delivered by the socially optimal drilling program (i.e., a program in which drilling is not distorted by lease terms). In each case, the owner's value is maximized with a finite primary term, and the optimal primary term is a decreasing function of the initial gas price. 

% OWNER VALUE AGAINST PRIMARY TERM AND DRILLING SUBSIDY
\begin{figure}[!t]
\begin{center}
\figcapt[\textwidth]{Expected value to the mineral owner as a function of lease terms}
\mbox{\subfloat[Owner's value vs. primary term]{\figinpt{width=.47\textwidth,clip}{simulations/EVlessorprofit_vs_priterm.pdf}}}
\mbox{\subfloat[Owner's value vs. drilling subsidy]{\figinpt{width=.47\textwidth,clip}{simulations/EVlessorprofit_vs_lc.pdf}}}
\fignote[\textwidth]{Note: In both panels, the royalty is held fixed at       53\%\unskip. In panel (a) the drilling subsidy is \$0, and in panel (b) the primary term is infinite. All expected values are shown as a percentage of the expected total surplus from the socially optimal drilling program. See text for details.}
\label{fig:owner_priterm_1well}
\end{center}
\end{figure}

The owner's optimal royalty of       53\%is substantially higher than the 20--25\% royalty rates typically observed in the Haynesville Shale. We believe that there are at least four factors driving our optimal royalty rate higher. First, the optimal royalty rate is closely related to the mineral owner's uncertainty about productivity, which is governed by the parameter $\sigma_{\theta}$. As discussed in section \ref{sec:Calibration}, our value of $\sigma_{\theta}$ is drawn from the dispersion of predicted production across the entire Haynesville Shale. If mineral owners have at least a rough idea of whether they are in a relatively productive or unproductive part of the shale, actual information asymmetry is less than what we are modeling, which will lead to a lower optimal royalty. We quantify this effect in sensitivity analyses, shown in table \ref{tab:sensitivity} of appendix \ref{appx:calibration}. When we reduce $\sigma_{\theta}$ by 50\% or by 75\%, we find that the optimal royalty falls from       53\%to       46\%or       38\%\unskip , respectively. We also find that with the lower optimal royalty, the optimal primary term is lengthened (to     5.50or     6.25years, respectively).

The second factor that may lead our model to predict higher-than-actual royalties is our assumption that owners can make a take-it-or-leave-it offer to firms. Leases in the Haynesville are instead typically negotiated, and if the firm has bargaining power the resulting royalty is likely to be less than what we find here (and the royalty should be zero if the firm has the ability to make a take-it-or-leave-it offer \citep{bib:demarzo}). Third, mineral owners may be risk averse relative to firms, which will lead them to prefer larger bonuses over larger royalties. Fourth, our optimal bonus calculations assume that there is only one firm to whom to offer a lease. In the presence of competition in the leasing market, bonus auctions become more effective at reducing firms' information rents, so that the optimal royalty is reduced.

% TOTAL SURPLUS AGAINST PRIMARY TERM
\begin{figure}[!t]
\begin{center}
\figcapt[\textwidth]{Sum of the owner's and firm's expected values as a function of lease terms}
\mbox{\subfloat[Firm + owner value vs. primary term]{\figinpt{width=.47\textwidth,clip}{simulations/EVtotal_vs_priterm.pdf}}}
\mbox{\subfloat[Firm + owner value vs. drilling subsidy]{\figinpt{width=.47\textwidth,clip}{simulations/EVtotal_vs_LC.pdf}}}
\fignote[\textwidth]{Note: In both panels, the royalty is held fixed at       53\%\unskip. In panel (a) the drilling subsidy is \$0, and in panel (b) the primary term is infinite. All expected values are shown as a percentage of the expected total surplus from the socially optimal drilling program. See text for details.}
\label{fig:total_priterm_1well}
\end{center}
\end{figure}

The analytic model presented in section \ref{sec:AnalyticModel} indicates that a drilling subsidy, rather than a primary term, should be combined with the royalty to maximize the owner's expected payoff. Panel (b) of figure \ref{fig:owner_priterm_1well} shows that an optimally-set drilling subsidy, combined with the       53\%royalty, indeed outperforms the optimal primary term. At an initial gas price of $\$     6.70 $\unskip /mmBtu, a lease with an optimally-set drilling subsidy of \input{single_numbers_tex/simresults/optlc.tex}million improves the owner's payoff by $\$      141 $\unskip ,000 relative to a lease with a royalty only; whereas a primary term improves the owner's payoff by $\$      104 $\unskip ,000 relative to a lease with a royalty only.\footnote{The \input{single_numbers_tex/simresults/optlc.tex}million drilling subsidy is optimal at the       53\%royalty associated with the optimal primary term. The optimal royalty and drilling subsidy combination is a       60\%royalty with a $\$     3.80 $million subsidy. This combination increases the owner's value by $\$      165 $\unskip ,000 relative to the       53\%royalty-only lease.}

Figure \ref{fig:total_priterm_1well} illustrates how the total (owner + firm) expected surplus from the lease is affected by the primary term (panel (a)) and drilling subsidy (panel (b)), holding the royalty fixed at       53\%\unskip. These results show that primary terms and drilling subsidies can increase not just the owner's value but also the total surplus from a lease. In fact, the primary term that maximizes total surplus at each initial price is nearly identical to the primary term that maximizes the owner's value in figure \ref{fig:owner_priterm_1well}. Thus, even though the bunching of drilling induced by primary terms appears highly distortionary ex-post, these results indicate that a primary term can increase both expected revenue and total surplus ex-ante, since it can partially mitigate the distortions induced by the royalty.

% OWNER AND TOTAL SURPLUS AGAINST ROYALTY
\begin{figure}[!t]
\begin{center}
\figcapt[\textwidth]{Owner's value and total surplus as a function of royalty}
\mbox{\subfloat[Owner's value vs. royalty]{\figinpt{width=.47\textwidth,clip}{simulations/EVlessorprofit_vs_royalty.pdf}}}
\mbox{\subfloat[Total surplus vs. royalty]{\figinpt{width=.47\textwidth,clip}{simulations/EVtotal_vs_royalty.pdf}}}
\fignote[\textwidth]{Note: In both panels, the primary term is held fixed at     4.75years, and the drilling subsidy is \$0. All expected values are shown as a percentage of the expected total surplus from the socially optimal drilling program. See text for details.}
\label{fig:owner_total_royalty}
\end{center}
\end{figure}

We also consider how expected values vary with the royalty. Panels (a) and (b) of figure \ref{fig:owner_total_royalty} display the owner's value and total surplus, respectively, as a function of the royalty, holding the primary term fixed at     4.75years. We see that both outcomes are highly sensitive to the royalty, and that total surplus is maximized at a royalty substantially lower than that which maximizes the owner's take.\footnote{The total-surplus-maximizing royalty is not zero because at zero royalty, the large optimal bonus causes many firm types to not accept the lease contract.} In particular, royalties that are too high cause excessive distortions to drilling (and to production conditional on drilling), and royalties that are too low leave the firm with large information rents. 

Figure \ref{fig:drillingprobs_priterm_1well}, panel (a) presents the per-period drilling probabilities associated with the social optimum, the royalty-only lease, and the lease with the optimal primary term. The royalty-only contract induces a significant delay to drilling. The contract with the primary term partially counteracts this delay but imposes a notch in the drilling profile upon expiration. The time path of drilling induced by the drilling subsidy, shown in panel (b), also partially counteracts the delay induced by the royalty and does so smoothly, without imposing a notch in the drilling profile.

% DRILLING PROBABILITIES IN THE PRESENCE OF PRIMARY TERM OR LESSOR COST
\begin{figure}[!t]
\begin{center}
\figcapt[\textwidth]{Drilling probabilities as a function of lease terms}
\mbox{\subfloat[Optimal primary term]{\figinpt{width=.47\textwidth,clip}{simulations/Edrillingprobs_priterm.pdf}}}
\mbox{\subfloat[Optimal drilling cost subsidy]{\figinpt{width=.47\textwidth,clip}{simulations/Edrillingprobs_LC.pdf}}}
\fignote[\textwidth]{Note: Drilling probabilities are evaluated at the time of lease signing, with a gas price at signing of $\$     6.70 $\unskip /mmBtu. See text for details.}
\label{fig:drillingprobs_priterm_1well}
\end{center}
\end{figure}

Finally, our model accounts for not only how lease terms affect drilling timing, but also for how they affect drilling inputs and production, conditional on drilling. In the absence of lease-induced distortions, the expected water input to a drilled well in our model is     12.5million gallons. But under the optimal royalty and primary term combination, expected water use conditional on drilling falls by       68\%to      4.0million gallons. This reduction in input use causes cumulative discounted lifetime production (conditional on drilling) to decrease by       19\%\unskip , from     3.54to     2.86million mmBtu.



\subsection{Multiple wells per lease}

Finally, we turn to the extended model in which each lease can accommodate multiple wells. Figure \ref{fig:owner_priterm_multwells}, panel (a) plots the mineral owner's expected value against the primary term length when leases accommodate        3wells, holding the royalty fixed at       53\%\unskip. Relative to the single-well lease case, shown in panel (a) of figure \ref{fig:owner_priterm_1well}, the optimal primary term is longer at all three initial prices and yields a smaller improvement in the owner's expected value. At the initial gas price of $\$     6.70 $\unskip /mmBtu, the optimal primary term is    12.50years and increases the owner's value, relative to an infinite primary term, by just      0.3\%\unskip. Figure \ref{fig:owner_priterm_multwells}, panel (b) shows that primary terms are also less effective at increasing total surplus than they were in the single-well case presented in figure \ref{fig:total_priterm_1well}.

% OWNER AND TOTAL VALUE VS PRIMARY TERM, MULTIPLE WELL UNITS
\begin{figure}[!t]
\begin{center}
\figcapt[\textwidth]{Owner's value and total surplus as a function of the primary term, with multiple-well leases}
\mbox{\subfloat[Owner's value vs. primary term]{\figinpt{width=.47\textwidth,clip}{simulations/EVlessorprofit_vs_priterm_multwells.pdf}}}
\mbox{\subfloat[Total surplus vs. primary term]{\figinpt{width=.47\textwidth,clip}{simulations/EVtotal_vs_priterm_multwells.pdf}}}
\fignote[\textwidth]{Note: In both panels, leases can accommodate up to        3wells, the royalty is held fixed at       53\%\unskip, and the drilling subsidy is \$0. Drilling only the first well is necessary to satisfy the primary term requirement and hold the lease indefinitely. All expected values are shown as a percentage of the expected total surplus from the socially optimal drilling program. See text for details.}
\label{fig:owner_priterm_multwells}
\end{center}
\end{figure}


There are two reasons why primary terms perform poorly for the mineral owner when leases accommodate multiple wells. First, the primary term has no effect on the firm's decision to drill any well other than the first one. These later wells therefore remain delayed in expectation due to the royalty. Second, the net cost of drilling a well at the deadline can be much lower than the bonus payment required to extend the lease, since the bonus is optimized to capture the value from drilling all of the potential wells, not just one of them. Thus, the primary term will substantially increase the probability of drilling the first well---potentially above and beyond the drilling probability at the social optimum---as shown in the simulated drilling probabilities and hazards in figure \ref{fig:drillingprobs_priterm_multwells}.

% MULTI WELL DRILLING PROBABILITIES AND HAZARDS
\begin{figure}[!t]
\begin{center}
\figcapt[0.9\textwidth]{First-well drilling probabilities and hazards as a function of lease terms, when the lease accommodates multiple wells}
\mbox{\subfloat[Drilling probabilities]{\figinpt{width=.47\textwidth,clip}{simulations/Edrillingprobs_priterm_multwells.pdf}}}
\mbox{\subfloat[Drilling hazards]{\figinpt{width=.47\textwidth,clip}{simulations/Edrillinghaz_priterm_multwells.pdf}}}
\fignote[0.9\textwidth]{Note: Drilling hazard each period is conditioned on not having drilled or let the lease expire prior to that period, as well as on accepting the initial lease terms (i.e., paying the bonus) in the first place. In all cases, leases can accommodate up to        3wells, and the drilling subsidy paid by the mineral owner is \$0. See text for details.}
\label{fig:drillingprobs_priterm_multwells}
\end{center}
\end{figure}

These results perhaps help explain some mineral owners' frustration at firms' ability to hold large amounts of acreage with a single well. In 2010, a group of mineral owners filed suit against the state of Louisiana, arguing that the DNR's 640-acre spacing rule---while perhaps appropriate for conventional, high-permeability formations in which a single gas well could drain such a large area---was excessive for the Haynesville shale, where multiple wells were required.\footnote{The plaintiffs did not prevail in court. \emph{Gatti et al. vs State of Louisiana}, decided January 15, 2014.} In other states, where pooling unitization practices tend to be governed more by common law than civil law, large mineral owners have moved to include aggressive ``retained acreage'' clauses in their leases that strictly limit the acreage that any one well can hold, though the precise specification of these clauses has also been the subject of litigation.\footnote{The increased use of retained acreage clauses was discussed by industry participants at the March, 2020 Rice University Workshop on Economic and Environmental Effects of Oil and Gas. These clauses are usually used in combination with a ``continuous development'' clause that allows the firm to retain all acreage in the lease so long as it is actively drilling and completing wells. See \url{https://www.oilandgaslawyerblog.com/more-on-retained-acreage-clauses/} for sample language that is recommended to mineral owners. For examples of recent litigation, see \citet{bib:durrett} and especially the discussion therein of \emph{Endeavor Energy Resources, L.P. v. Discovery Operating, Inc.}, decided by Texas's Eastland Court of Appeals in 2014 and by the Texas Supreme Court on April 13, 2018.}



\section{Conclusion \label{sec:Conclusion}}

This paper begins by presenting evidence that primary terms embedded into mineral leases in the Haynesville Shale in Louisiana have led to a substantial bunching distortion in firms' drilling activity. This result raises the question of why mineral owners and firms would include these apparently surplus-decreasing terms in their leases. The model we develop in this paper suggests an answer: primary terms can actually improve the owner's ex-ante expected value by providing a complement to the royalty. The royalty reduces firms' information rents, and the primary term then partially mitigates the delay distortion induced by the royalty.

Using a computational model that we calibrate to the Haynesville, we show that an optimally-set primary term can increase both the mineral owner's expected value and the total (owner + firm) expected surplus from the lease. These improvements occur despite the discontinuous drop in drilling probability that the contract induces at the primary term's expiration date. In line with our analytic results, we show that alternative contract designs that use a drilling subsidy rather than a primary term can achieve additional, albeit modest, improvements in the owner's expected take. 

We also show that conventional primary term clauses are less effective at delivering value to resource owners when either state regulation or lease language allows a single well to hold acreage sufficient to drill multiple follow-up wells in the future. This last result is consistent with recent litigation initiated by Louisiana mineral owners, and with owners' moves in other states to adopt stricter retained acreage clauses.

This paper's model and results could be enriched or extended in several ways in future work. Our model abstracts away from factors such as learning and economies of scale that impact firms' drilling and completion productivity \citep{bib:covert,bib:agerton,bib:fetter_etal,bib:hodgson,bib:steck}. Integrating these productivity dynamics into our model would be computationally challenging but could reveal implications for oil and gas lease design. Our model also concentrates all bargaining power into the hands of the mineral owner. This assumption is useful for revealing the contractual form that is most advantageous for the owner but in practice is violated in private mineral leasing, where terms are often bilaterally negotiated. Theoretical advances in modeling bargaining problems with asymmetric information that go beyond cases in which either the seller or the buyer (as in \cite{bib:demarzo}) can make take-it-or-leave-it offers would be valuable in improving our understanding of private oil and gas leasing outcomes. Finally, our approach could be combined with the modeling in \citet{bib:cong} to evaluate the economics of leases with primary terms when the timing with which leases are initially signed is endogenous.

Extending our modeling framework to other settings is also likely to be worthwhile. Within the oil and gas sector, expanding the geographic scope to include the other major U.S. shale plays would permit an assessment of how royalties and primary terms have affected aggregate U.S. oil and gas supply. Such work could examine the role of mineral leases in driving misallocation of shale drilling across time and space in the U.S., relating to \cites{bib:askeretal} recent work on aggregate wedges between optimal and observed global oil extraction, and to \cites{bib:gilje} documentation of how debt renegotiations have distorted U.S. shale drilling. Our framework could also be used to evaluate the economics of carbon policies in a second-best environment in which oil and gas production is already distorted by mineral lease terms. Finally, the ideas in this paper could be extended to other settings---such as retail franchising or intellectual property licensing---in which principals sell time-limited development options to agents.



\bibliography{References}

\singlespace
\newpage


\appendix
\setcounter{page}{1}
\renewcommand{\thepage}{A-\arabic{page}}

\section*{Online Appendix for ``The Economics of Time-Limited Development Options: The Case of Oil and Gas Leases''}




\section{Detail on data sources, cleaning, and merging} \label{appx:data}

In this data appendix, we discuss: (1) our data sources; (2) how we estimate well decline and the present value of well cumulative production; (3) how we clean lease data and match leases to units; and (4) how we match wells to Haynesville units.

\subsection{Data sources}

We gather data from the following sources:
\begin{itemize}
\item Publicly-available Louisiana DNR Strategic Online Natural Resources Information System (SONRIS) data on well drilling and completions. The well-level data include spud date, completion date, well name, formation targeted, top hole location, bottom hole location, and lateral location shapefiles. DNR SONRIS includes  a data file that lists Haynesville wells. DNR SONRIS also includes a shapefile for Haynesville units.
\begin{itemize}
\item Drilling cost and fracking input data. We obtain drilling cost information from reports (``Applications for Well Status Determination'') that unit operators file with the Louisiana DNR for the purpose of determining severance taxes. We obtain data on fracking inputs (water use and the number of frac stages used) from well completion reports that are available from the DNR. We used manual double-entry to digitize this information from the raw pdf files.
\end{itemize}
\item Enverus production data. Enverus takes unit-level reported monthly production data from the Louisiana DNR and then imputes well-level monthly production using the start date of each well's production.
\item Enverus lease data. Enverus collects data on leases signed in Louisiana. Further details are below.
\item Henry Hub natural gas futures prices. We obtained daily futures price data, at all available delivery dates, from Bloomberg.
\item Enverus dayrate data. We use dayrates that correspond to the ``ArkLaTx'' region, for rigs with depth ratings between 10,000 and 12,999 feet (which corresponds to the depth of the Haynesville).
\end{itemize}


\subsection{Well data and production decline estimation \label{appx:decline}}

We identify Haynesville wells using three sources. First, for each well we see if the well is included in an auxiliary DNR file that is limited to Haynesville wells. Second, we use the name of the well: wells targeting the Haynesville typically have names that begin with ``HA'' or ``HAY''. Third, we check whether the listed formation that the well targets is the Haynesville. We denote a well as a Haynesville well if it satisfies at least one of these three criteria. We also impose a restriction that Haynesville wells must have been spudded on or after September 2006.

To estimate wells' production decline, we follow \citet{bib:patzek}, which derives decline curves for shale gas formations. This paper shows that production initially declines inversely proportional to the square root of time, and then begins to decline more quickly, at an exponential rate, once the well's fractures interfere with one another. More precisely, we assume that cumulative production of natural gas for well $i$ at month $t$ takes the following functional form:

\begin{equation}
m_i(t) =
\begin{cases}
M_i \sqrt{t / \tau } & \mbox{if } 0 \leq t \leq \tau \\
M_i +  \dfrac{ M_i }{2 \tau d} \left[ 1 - \exp(-d (t - \tau)) \right] & \mbox{if } t > \tau
\end{cases}
\end{equation}

where $\tau$ is the time at which the decline function changes to exponential, $d$ is the exponential decline rate, and $M_i$ is a well-specific production multiplier corresponding to the expected cumulative production at $t = \tau$.\footnote{This functional form incorporates the structure of \citet{bib:patzek} but also imposes the requirement that cumulative production be differentiable at $t=\tau$.}

Before estimating these parameters, we make a number of adjustments to the data. First, because a well's production is substantially affected by the length of the lateral well leg, we normalize the measure of cumulative production by a scalar $s_i$ which is equal to 1485 meters divided by the length of the lateral portion of the well. We drop any well with missing lateral length information or a well lateral of less than 150 meters to eliminate potentially misclassified vertical wells.\footnote{150 meters is a reasonable breakpoint: there are no wells with laterals between 150 and 400 meters in our data.}

We find that about 7
\unskip \% of wells had recompletions. As recompletions are designed to rapidly increase production, we exclude from the data observations that come during or after months in which well recompletions were performed. (When we later use our estimates to predict total production, we assume no recompletions.)

Following \citet{bib:patzek}, we limit the sample to observations that are the fourth month or later ($t \geq 4$) because early months of production tend to be noisy. This noise is due in part to the fact that hydrofracturing water is still being back-produced in the early months of production. Similarly, for any given date with no production but in which there is production both before and after the date, we assume that the production process is paused on that date and resumes when production resumes.

Rather than estimating $\tau$ directly, we use estimates from \cite{bib:male} for the Haynesville, which finds that $\tau$ is 14.16 months. We then use non-linear least squares to find the values of $M_i$ and $d$ that minimize the sum of the squared differences between true and predicted log cumulative production, as shown in equation (\ref{eq:decline_nonlinear_OLS}).

\begin{equation}
\sum_i  \sum_{t | t \geq 4} \left( \log m_i(t) s_i - \log \widehat{m}_i(t | M_i, \tau, d) \right)^2
\label{eq:decline_nonlinear_OLS}
\end{equation}

The estimated decline parameter $d$ is equal to    0.037\unskip. The 25th, 50th, and 75th percentiles of the estimated $M_i$ are     1.51\unskip\space million,     2.02\unskip\space million, and     2.53\unskip\space million mmBtu, respectively.

We then use our estimates of $d$ and $M_i$ to predict total discounted well production (equation (\ref{eq:impute_total_PV_production})). Following \citet{bib:gulen}, we assume that wells have a total production lifetime of 20 years. We use an annual discount factor of    0.909\unskip, following \cite{bib:kellogg}. Figure~\ref{fig:maps_productivity}, panel (a) shows a map of our measures of the present value of total well production. Where there are multiple wells, we take an average over all wells within the unit. Units with no drilling have no shading and are labeled NA.

\begin{equation}
Y_i = \sum_{t = 1}^{240} [\widehat{m}_i(t | M_i, \tau, d) - \widehat{m}_i(t-1 | M_i, \tau, d) ] \delta^{t-1}
\label{eq:impute_total_PV_production}
\end{equation}


\subsection{Lease data and clustering of duplicate leases within units} \label{appx:lease_data}

We downloaded, in February, 2016, raw data of oil and gas leases in Louisiana from Enverus. We keep only leases in Bienville, Bossier, Caddo, De Soto, Natchitoches, Red River, Sabine, and Webster parishes---the parishes that cover the Louisiana portion of the Haynesville formation. Because we ultimately map leases to units, we keep only those observations that report Public Land Survey System township, range and section.

We keep observations that are listed as being leases, memo of leases, lease options, lease extensions, and lease amendments. We drop observations that are mineral rights assignments, lease ratifications, mineral deeds, royalty deeds, and other documents. Leases include information on the grantor of the lease (typically the original mineral owner) and the grantee (the oil and gas firm that leases the land). In some cases we find that oil and gas firms are listed as grantors, with other oil and gas firms listed as grantees. As these observations are likely cases where the land was re-leased or subleased, we drop these observations from our sample.

We drop leases with zero or missing acreage. We also drop excess lease observations that are perfect duplicates, leases that have lengths of fewer than 10 days, and leases in which the reported township, range, and section are not within the stated reported parish.

We find that in some cases, a single firm grantee has leased from multiple grantors, and the reported acreage appears to be the total over all grantors. We identify these leases by identifying duplicates that share the same grantee name and the same acreage, and where the acreage reported is unusual---i.e., is either large and/or is not equal to a multiple of common lot sizes (e.g., 10 acres, 40 acres). In these cases we impute a new acreage measure by dividing the reported acreage by the number of apparent duplicates.

After taking the above steps, we find that the total leased acreage in a unit still sometimes adds up to more than the total acreage of the unit (usually 640 acres), and sometimes significantly so. Many of these cases appear to be driven by undivided mineral interests: cases where there are multiple grantors on the same plot (e.g., husband and wife, multiple siblings or cousins, etc), and separate observations for each grantor. In other cases it appears that data were entered multiple times and inconsistencies were not reconciled, so that the excess observations were not dropped when we removed duplicates.

To identify these likely duplicates, we use an agglomerative, hierarchical clustering method described by \url{https://stat.ethz.ch/R-manual/R-devel/library/stats/html/hclust.html}. In particular, we use the hclust function within the cluster package, version 2.0.7-1, for R. The hclust function uses information on how similar multiple observations are to each other to determine whether they are likely duplicates. The algorithm puts observations that are likely duplicates into the same ``cluster''; from there we use proportional downweighting of all observations within the same cluster to obtain updated acreage. This method relies on constructing some kind of measure of similarity between any two observations i and j. Depending on the threshold level of similarity that the researcher imposes, the number of clusters can range from the total number of observations (no clustering) to 1 (all observations are placed within the same cluster).

This similarity measure we use is a Euclidean-like distance measure in which the distance between observation $i$ and observation $j$ takes the form:

\begin{equation}
d_{ij} = \sqrt{\sum_k{w_k m_k(x_i^k, x_j^k)}}
\end{equation}

Here $k$ indexes characteristics of the observation---e.g., grantor name, the start date of the lease, the acreage, the reported royalty rate, etc. The function $m_k$ is a function that determines how similar two observations are, and is equal to 0 if identical, and positive otherwise. Depending on the characteristic, we use different types of $m_k$ functions:

\begin{itemize}
\item $m_k(x_i^k,x_j^k) = (x_i^k-x_j^k)^2$ for some numerical characteristics like the start date of the lease. Prior to inputting variables $x^k$ into this function, we standardize them so that they have a mean of zero and standard deviation of one.
\item $m_k(x_i^k,x_j^k) = 1(x_i^k\neq x_j^k)$ for other numerical and binary characteristics like reported royalty rate, acreage, and whether there is an extension option.
\item $m_k(x_i^k,x_j^k)$ is a fuzzy match score for string characteristics like grantee name and grantor name. We use the partial\_ratio function from the fuzzywuzzy Python package, version 0.16.0. The partial\_ratio function uses Levenshtein distance augmented with partial string matching. It allows us to identify cases where some subsets of words within strings match or nearly match, even if the length of the two strings is very different. This technique is useful for catching cases with identical last names but differing or missing first names. We scale this measure so that it ranges from 0 to 1.
\end{itemize}

For cases where information is missing, we set a value of $m_k = 0.4$ if both observations are missing and $m_k = 0.7$ if only one observation is missing.

$w_k$ are positive weights. We set $w_k = 1$ for all characteristics other than acreage, for which we set $w_k = 100$.\footnote{In an alternative specification, we set $w_k = 1/2$ for grantor name and $w_k = 2$ for grantee name. This specification is motivated by the facts that grantor names can potentially differ substantially for siblings if they change their names through marriage, and that there is little variation in grantee name within units. Results using this approach are similar to what we obtain from our baseline specification with equal weights on grantor and grantee names.} This weighting ensures that leases that vary in acreage will not be presumed to be duplicates.

How many observations are clustered together depends on the threshold level of similarity imposed by the researcher. To determine our threshold, we choose a calibration date of January 1, 2010, examining only the leases that were active on that date.\footnote{We find that using other calibration dates gives similar results. We use January 1, 2010, as it was at a period of peak leasing, and therefore a period in which it is most likely that most of a section had been leased. Leases whose primary terms would have expired but may have been extended are not included in this group.} We first examine every possible threshold that could be used to cluster the leases in each unit. For each possible threshold, we find the resulting clusters, downweight each lease's acreage by the total number of observations in its cluster, and then compute what total leased acreage would be within the unit. Then, for each unit, we find the threshold would be that would set total acreage leased to be equal to or just less than the total unit area. We refer to this threshold as the unit-level threshold height. We then set our preferred overall threshold to be equal to the 90th percentile of all the unit-level thresholds.\footnote{We find similar results if we use a threshold height of 85\%. We find that using the 100th percentile is not feasible as some outlier units have leases with large acreage such that total leased acreage is always greater than unit acreage unless leases of different acreage are combined into the same cluster.} The threshold height that results from this computation is~2.004
\unskip. 

We then apply the clustering procedure, using this threshold, to each unit and each quarterly date of our sample, ranging from January 1, 2005 to January 1, 2016. This procedure gives us lease by date-specific downweights. We find that in some cases lease downweights vary depending on the date. For example, a lease may be in a cluster of five on April 1, 2010 but a cluster of six July 1, 2010---resulting in a downweight of 1/5 for April 1, 2010 and a downweight of 1/6 for July 1, 2010. In these situations, we take the inverse of the arithmetic average of the inverse downweight over all quarterly dates to obtain a master downweight for each lease (yielding, in this example, a weight of 1/5.5).

In some outlier unit-quarters we find that even with this downweighting, total leased acreage still exceeds section acreage. In these cases we then proportionally reduce the area of all leases in the unit so that total leased acreage is equal to total unit acreage in the most heavily leased quarter.


\subsection{Matching of wells to units} \label{appx:wellunitmatch}

To match wells to units, we use information on the reported laterals, reported bottom holes, and reported top hole locations. If, for a given well, the data only report top hole location, we use the location of the top hole to identify which unit the well is in. If the data report bottom hole but not lateral information, we use the location of the bottom hole to identify which unit the well is in. If the data report lateral information, we use the unit that the lateral runs through to identify the well's unit.

In a few instances, the well lateral intersects multiple units. There are two possible reasons for these occurrences. One is that the well's top hole is located in a different unit than the unit the well extracts from, for the purpose of sharing a well pad with other wells or to give sufficient space to accommodate the curvature of transitioning from the vertical to the horizontal while still extracting from a maximum area within the targeted unit. A second reason is that the well actually targets multiple units. In cases where a well bore passes through multiple units, we only match a well to a unit if at least 300 meters of the horizontal well bore pass through the unit.

\section{Additional empirical analysis} \label{appx:more_data_analysis}

This appendix presents additional empirical results relating to the bunching analysis presented in section \ref{sec:bunching}.



\subsection{Bunching analysis} \label{appx:bunching}

To test the statistical significance of the drilling bunching shown in figure \ref{fig:drillprob_first_spud_vs_first_lease_expire}, we use a bunching estimator similar to that of \citet{bib:chetty}. We take time of spud relative to first lease expiration date, discretize it to the quarterly level, and compute total wells spudded (across all units in our analysis sample) for each quarter (34 quarters in total). We create some indicator variables for whether the spud date is two quarters before lease expiration (pre\_2), one quarter before lease expiration (pre\_1), one quarter after lease expiration (post\_1), and two quarters after lease expiration (post\_2). We also add similar variables for spud timing relative to the extension expiration date (pre\_ext2, pre\_ext1, post\_ext1, and post\_ext2).

We then estimate a regression of the form:
\begin{align}
\begin{split}
c_t & =  {f}(t) + \beta_1 \cdot \mbox{pre\_2} + \beta_2 \cdot \mbox{pre\_1} + \beta_3 \cdot \mbox{post\_1} + \beta_4 \cdot \mbox{post\_2}  \\
&  + \beta_5 \cdot \mbox{pre\_ext2} + \beta_6 \cdot \mbox{pre\_ext1} + \beta_7 \cdot \mbox{post\_ext1} + \beta_8 \cdot \mbox{post\_ext2} + \varepsilon_t
\end{split}
\label{eq:bunching}
\end{align}

\noindent where $c_t$ is total well count, $t$ is quarter, and ${f}(t)$ is a polynomial of degree 9. Our main regression estimates are in column (1) of table \ref{tab:bunching}. The estimates of $\beta_1$, $\beta_2$, $\beta_5$, and $\beta_6$ are all statistically significant with p-values less than 0.05, indicating that there is significantly more drilling in the two quarters prior to the primary term expiration and the two quarters prior to any extension term expiration. Column (2) of table \ref{tab:bunching} uses the same empirical specification as column (1) except that the dependent variable is the log of the count rather than the count, and results are similar. For a sense of magnitude, the estimate of $\beta_2$ (the coefficient on ``pre\_1'') in column (2) means that the actual number of wells drilled is 1.1log points larger than the polynomial fit in the quarter prior to expiration. In figure \ref{fig:bunching}, we plot our data and the number of wells predicted by our polynomial fit, which graphically displays the size of the bunching effect.

% BUNCHING ESTIMATE TABLE
\begin{table}[!t]
\centering
\figcapt{Bunching estimates}
\begin{tabular}{l*{6}{c}} \hline\hline
                &\multicolumn{1}{c}{count }&\multicolumn{1}{c}{log(count)}&\multicolumn{1}{c}{ count }&\multicolumn{1}{c}{log(count)}&\multicolumn{1}{c}{ count }&\multicolumn{1}{c}{log(count)}\\
\hline
pre\_2           &    12.72&     0.40&     1.06&     0.19&     1.02&     0.21\\
                &   (5.25)&   (0.21)&   (0.88)&   (0.30)&   (0.82)&   (0.30)\\
pre\_1           &    60.94&     1.10&     3.91&     0.78&     3.99&     0.88\\
                &   (5.71)&   (0.21)&   (1.36)&   (0.29)&   (1.29)&   (0.29)\\
post\_1          &     6.03&     0.31&     0.31&    -0.07&     0.44&     0.04\\
                &   (5.74)&   (0.21)&   (0.77)&   (0.28)&   (0.70)&   (0.27)\\
post\_2          &    -8.92&    -0.09&    -0.50&    -0.21&    -0.53&    -0.17\\
                &   (5.32)&   (0.19)&   (0.61)&   (0.25)&   (0.58)&   (0.25)\\
pre\_ext2        &    13.70&     0.26&     1.24&     0.52&     1.39&     0.53\\
                &   (4.75)&   (0.19)&   (1.11)&   (0.34)&   (1.05)&   (0.34)\\
pre\_ext1        &    21.73&     0.52&     1.79&     1.19&     1.93&     1.15\\
                &   (5.23)&   (0.22)&   (1.26)&   (0.31)&   (1.16)&   (0.31)\\
post\_ext1       &    -4.99&    -0.24&    -0.32&    -0.05&    -0.23&     0.04\\
                &   (5.19)&   (0.21)&   (0.72)&   (0.32)&   (0.72)&   (0.32)\\
post\_ext2       &    -3.70&    -0.11&    -0.26&     0.28&    -0.15&     0.41\\
                &   (4.61)&   (0.17)&   (0.64)&   (0.30)&   (0.67)&   (0.37)\\
\hline
Quarter of lease &        X&        X&         &         &         &         \\
Quarter by quarter of lease &         &         &        X&        X&        X&        X\\
Fixed effects   &         &         &         &         &        X&        X\\
R Squared       &     0.97&     0.93&     0.24&      0.3&     0.42&     0.44\\
Observations    &       35&       30&      363&      235&      363&      235\\
\hline\hline
\end{tabular}

\fignote[\textwidth]{Note: Table presents estimates of equation (\ref{eq:bunching}). Newey-West standard errors, computed with two quarterly lags, are in parentheses. Estimates in columns (1) and (2) use data that are aggregated to the lease-level quarter, which is defined as the time between first primary term expiration and spudding, measured at quarterly intervals. Estimates in columns (3) through (6) use data that is aggregated to the lease-level quarter by calendar-level quarter. That is, these columns aggregate wells drilled that share both a common lease-level quarter and a common calendar quarter-of-sample. Estimates in columns (5) and (6) include calendar quarter fixed effects.}
\label{tab:bunching}
\end{table}

% BUNCHING ESTIMATE FIGURE
\begin{figure}[!t]
\centering
\figcapt[\textwidth]{Estimates from bunching analysis}
\includegraphics[scale=0.8]{section_descript/drill_bunching_allsections.pdf}
\fignote[0.8\textwidth]{Note: Figure presents data and estimates corresponding to the bunching analysis in column (1) of table \ref{tab:bunching}. Plotted data include counts of wells spudded, the quarters to which we add bunching fixed effects, and the polynomial predicted probabilities given the bunching estimator fixed effects. Timing is relative to the expiration date of the first lease within the unit to expire. Vertical lines are drawn at the date of first lease expiration and two years after first lease expiration.}
\label{fig:bunching}
\end{figure}

We also explore the possibility that these high spikes in drilling are being driven by other factors that might affect drilling in particular time periods. For instance, one might worry that periods with substantial lease expirations might coincide with periods in which gas prices or industry-wide productivity is high. To address this possibility, we construct a measure of total spud counts at the calendar quarter by quarter of lease level. For example, one observation in this count data will be the total number of spuds in 2010 quarter 3 when the spud happened between 3 and 6 months before the first primary term is set to expire. Columns (3) and (4) of table \ref{tab:bunching} use the same empirical specification as columns (1) and (2), only with this more disaggregated data. Columns (5) and (6) then use the same disaggregation as columns (3) and (4), but also add in calendar-time quarter fixed effects. Across columns (3) and (5), the coefficient on pre\_1 is large, statistically significant, and similar in magnitude, implying that the high drilling before the expiration date is not being driven by high drilling at particular calendar dates. The same holds in logs for columns (4) and (6).


\subsection{Drilling inputs and outputs as a function of drilling time} \label{appx:drlginputs}

This appendix examines how wells' natural gas production, water inputs, and reported drilling costs vary depending on whether drilling is done early, just prior to the first primary term expiring, or after the first primary term expires. Figure \ref{fig:scatter}, panel (a) plots the logged first 12 months of each unit's first well's gas production against the date of drilling relative to the date that the first primary term expires. Panels (b) and (c) present similar information, but with water use and reported drilling costs as the outcome variables, respectively. Over all three plots, the data are noisy, and there is no strong relationship between the outcome variables and drilling timing, apart from a slight downward trend over time. In particular, wells drilled just before first lease expiration are not especially unproductive and do not use especially low levels of inputs.

% SCATTER PLOTS
\begin{figure}[!t]
\figcapt{Wells' production, water use, and cost vs. time relative to first lease expiration}
\mbox{\subfloat[First 12 months of gas production (log mmBtu)]{\figinpt{width=.47\textwidth,clip}{section_descript/scatter_lpoly_log_ProdC_first12_gas.pdf}}}
\mbox{\subfloat[Water use (log gallons)]{\figinpt{width=.47\textwidth,clip}{section_descript/scatter_lpoly_log_water_RA2.pdf}}}
\mbox{\subfloat[Reported drilling and completion cost (log \$2014)]{\figinpt{width=.47\textwidth,clip}{section_descript/scatter_lpoly_log_well_cost_RA1.pdf}}}
\fignote[0.9\textwidth]{Note: Data are for the first well drilled in each unit. The horizontal axis is measured in days relative to the first lease expiration that the well was spudded. The line is the predicted value from a local polynomial regression.}
\label{fig:scatter}
\end{figure}



\subsection{Distribution of the expiration dates of leases} \label{appx:expirationdist}

The vast majority of units have multiple leases on them. Typically, leases are signed at different dates, and therefore the primary terms tend to end on different dates. In this subsection, we show that the distribution of the expiration dates affects the extent to which drilling responds to the expiration of the first lease in a unit.

For each unit, we calculate the amount of time from when the first lease expires to the time at which 50\% of all acreage would expire. We refer to this duration as time-to-50\%-expiration. We then compare drilling for units with above-median time-to-50\%-expiration to units with below-median time-to-50\%-expiration. Figure \ref{fig:longtime_shorttime_50percent_leasing} shows that units with above-median time-to-50\%-expiration have a less-pronounced spike in drilling in the few months prior to the first lease expiring. This result is consistent with the theory that if most of the leases in a unit do not expire for some time after the first lease expires, there is a reduced incentive for the firm to drill prior to the first expiration.

% FIGURE COMPARING DRILLING IN UNITS WITH ABOVE VS BELOW MEDIAN TIME TO 50% EXPIRATION
\begin{figure}[!t]
\centering
\figcapt[\textwidth]{Comparison of units with more vs less time before most acreage expires}
\includegraphics[scale=.6]{section_descript/drillprob_by_lease_length_leaseExpireFirst_first_spud.pdf}
\fignote[0.9\textwidth]{Note: Figure shows kernel-smoothed estimates of the probability of drilling the first Haynesville well in a unit on a given date, relative to the expiration date of the first lease within the unit to expire. Vertical lines are drawn at the date of first lease expiration and two years after first lease expiration. ``Longer (shorter) time to 50\% of leases expiring'' corresponds to units with an above (below)-median time between when the first lease expires and 50\% of all acreage expires.}
\label{fig:longtime_shorttime_50percent_leasing}
\end{figure}



\subsection{Information externalities and common pools} \label{appx:infoexternality}

\begin{figure}[!t]
\centering
\figcapt[\textwidth]{Comparison of units with identical vs different neighboring operators}
\mbox{\subfloat[Units in analysis sample]{\figinpt{width=.47\textwidth,clip}{section_descript/drillprob_neighb_50p_1p2.pdf}}}
\mbox{\subfloat[All Haynesville units]{\figinpt{width=.47\textwidth,clip}{section_descript/drillprob_neighb_50p_1p2_allunits.pdf}}}
\fignote[0.9\textwidth]{Note: Figure shows kernel-smoothed estimates of the probability of drilling the first Haynesville well in a unit on a given date, relative to the expiration date of the first lease within the unit to expire. Vertical lines are drawn at the date of first lease expiration and two years after first lease expiration. Figure compares units in which most of the nearby units have the same operator vs. units where most of the nearby units have a different operator. Neighboring units are defined as those with centroids within 1.2 miles of the centroid of the given unit. Panel (a) limits the units to our analysis sample, as described in subsection \ref{sec:poolingunitdata}. Panel (b) uses all Haynesville units.}
\label{fig:neighbor_50_1p2}
\end{figure}

One alternative explanation for the drilling patterns we observe is a cross-unit information or common pool externality. Common pool externalities might lead to a race to drill. Information externalities might lead to a reluctance to drill, as firms wait to learn from other firms' drilling outcomes. To examine whether such externalities might be driving our results, we compare units in which most of the unit's neighbors are operated by the same operator with units in which most of the neighbors are operated by a different operator. The former should be less susceptible to externalities than the latter. We define neighboring units as those for which the centroids of the two units are within 1.2 miles (results are similar when using a threshold of 1.7 miles, which will include the diagonal units).

In figure \ref{fig:neighbor_50_1p2}, panel (a) we compare drilling probabilities for these two types of units, using our analysis sample described in subsection \ref{sec:poolingunitdata}. For both types of units, we observe substantial bunching of drilling just before the first lease in the unit expires, leading us to conclude that cross-unit externalities are not a substantial driver of our results. Because the analysis sample has few units in which $\geq50\%$ of neighboring operators are the same, panel (b) uses the full set of Haynesville units that had drilling. It finds similar results.



\section{Derivation of the optimal contingent contract for the mineral owner \label{appx:optcontract}}

This appendix derives the optimal contract for the principal (mineral owner), given the model setup from section~\ref{sec:Setup}. The exposition below closely follows parts of \citet{bib:laffonttirole1986} and \citet{bib:board}. To facilitate the derivation of the optimal contract, we follow the standard approach of considering a direct revelation mechanism in which the firm reports a type $\hat{\theta}$ and is then assigned an up-front ``bonus'' transfer of $R(\hat{\theta})$ at $t=0$ and a contingent payment $z_t(\hat{\theta},y,P^t)$ to be paid when the option is executed. For now, we allow this payment to be contingent on the reported type, ex-post production, and the price history up to execution, though in practice conditioning only on the first two arguments and the price at execution will be necessary for optimality.


\subsection{Firm's problem \label{appx:FirmProb}}

The firm must make three decisions, in sequence:
\begin{enumerate}
\item Report a type $\hat{\theta}$ to the owner at $t=0$ (or opt-out)
\item Choose a time $\tau\in\{1,...T\}$ at which to exercise the option to drill, where $\tau=\infty$ signifies not drilling.
\item Conditional on drilling, select an effort level $e\in\mathbb{R}^+$
\end{enumerate}

Let $\tau^*(\theta,z)$ denote a decision rule that dictates whether the well should be drilled in each period $t$ given the gas price $P_t$ (we suppress the dependence of $z$ on $\hat{\theta}$, $y$, and $P^t$). The firm's problem, conditional on participation, is then given by
\begin{align}
\max_{\hat{\theta},\tau^*(\theta,z),e}\Pi(\hat{\theta},\tau^*(\theta,z),e,\theta)&=E_P[(P_\tau g(e)\theta-(c_0+c_1e) \nonumber \\
&-z_\tau(\hat{\theta},g(e)\theta(1+\varepsilon),P^t))\delta^\tau]-R(\hat{\theta}), \label{eq:firmprob}
\end{align}

where $E_P$ is the expectation at the start of period 0, taken over all prices. Note that total surplus is maximized by the solution to (\ref{eq:firmprob}) when the transfers $z$ and $R$ are set to zero.

From the definition of $g(e)$ in section~\ref{sec:Setup}, it is clear that the effort selection problem has a unique, interior solution. In addition, the decision rule $\tau^*(\theta,z)$ will be given by an optimal stopping rule.\footnote{\citet{bib:board} proves the existence of such a rule for the case in which effort $e$ is fixed. Existence in our model follows the same proof, with the assumptions that $P_t$ and $g(e)$ are bounded above replacing the \citet{bib:board} assumption that costs are bounded below.}

We restrict attention to truth-telling mechanisms that induce the agent to report $\hat{\theta}=\theta$. Let $\tau(\hat{\theta})$ and $e(\hat{\theta})$ denote the timing rule and effort function that correspond to the optimal truthful mechanism. Because drilling is observable, $\tau(\hat{\theta})$ can be imposed by the owner. For truth-telling to be incentive compatible, it must be the case that $e(\hat{\theta})$ (which is not contractible and therefore cannot be imposed) is the optimal effort level for the firm, subject to the mechanism.

To characterize the firm's ability to deviate from $e(\hat{\theta})$ and thereby reap information rent, we follow \citet{bib:laffonttirole1986} by first restricting our attention to deviations in a {\it concealment set} in which, for any report $\hat{\theta}$, the chosen effort $\tilde{e}$ is such that $g(\tilde{e})\theta=g(e(\hat{\theta}))\hat{\theta}$. Thus, absent uncertainty generated by $\varepsilon$, any deviation outside the concealment set can be detected by the owner.\footnote{In the presence of a non-degenerate distribution $\Lambda(\varepsilon)$, the sufficiency condition for implementing the mechanism will be stricter than the conditions given below that the optimal stopping time is decreasing in $\theta$ and that $\partial e(\theta)/\partial\theta\geq0$ (also see footnote \ref{fn:affinesuf}).}

Within the concealment set, the firm's choice of report $\hat{\theta}$ determines the firm's effort level $\tilde{e}$. Define an inverse production function $H(E)$ by $g(H(E))=E$. The derivatives of $g$ and $H$ are related by $H'(g(e))=1/g'(e)$. The firm's problem may then be written:
\begin{align}
\max_{\hat{\theta}}\Pi(\hat{\theta},\theta)&=E_P[(P_{\tau(\hat{\theta})} g(e(\hat{\theta}))\hat{\theta}-(c_0+c_1H(g(e(\hat{\theta}))\frac{\hat{\theta}}{\theta})) \nonumber \\
&-z_{\tau(\hat{\theta})}(\hat{\theta},g(e(\hat{\theta}))\hat{\theta}(1+\varepsilon),P^{\tau(\hat{\theta})}))\delta^{\tau(\hat{\theta})}]-R(\hat{\theta}). \label{eq:firmrestricted}
\end{align}

To obtain the marginal information rent for a firm of type $\theta$, we need the derivative of $\Pi(\hat{\theta},\theta)$ with respect to $\theta$. Using the envelope theorem, this marginal rent is given by the partial derivative:\footnote{Following \citet{bib:board}, we use generalised envelope theorem of \citet{bib:milgromsegal} because the space of stopping times is more complicated than $\mathbb{R}^+$.}
\begin{equation}
\frac{\partial\Pi(\hat{\theta},\theta)}{\partial\theta}\bigg|_{\hat{\theta}=\theta}=
E_P\left[\frac{c_1g(e(\theta))\delta^{\tau(\theta)}}{g'(e(\theta))\theta}\right] \label{eq:PartInfoRent}
\end{equation}

Equation (\ref{eq:PartInfoRent}) is the first-order incentive compatibility condition. The second order monotonicity condition is given by
\begin{equation}
\frac{\partial\Pi(\hat{\theta},\theta)}{\partial\theta\partial\hat{\theta}}\geq0. \label{eq:ASOC_SOC}
\end{equation}

From equation (\ref{eq:PartInfoRent}), it is apparent that the mechanism will satisfy this condition if the optimal stopping time is decreasing in $\theta$ and $\partial e(\theta)/\partial\theta\geq0$.

To see that satisfaction of equation (\ref{eq:ASOC_SOC}) is sufficient for incentive compatibility, consider a type $\theta$ firm that deviates to report type $\hat{\theta}\neq\theta$. We have:
\begin{align}
\Pi(\hat{\theta},\theta) &= \Pi(\hat{\theta},\hat{\theta})+\int_{\hat{\theta}}^\theta \frac{\partial\Pi(\hat{\theta},s)}{\partial\theta}ds \nonumber \\
&\leq \Pi(\hat{\theta},\hat{\theta})+\int_{\hat{\theta}}^\theta \frac{\partial\Pi(s,s)}{\partial\theta}ds \nonumber \\
&= \Pi(\theta,\theta),
\end{align}

where the second line follows from the monotonicity condition.

Given incentive compatibility, integration of equation (\ref{eq:PartInfoRent}) yields the firm's information rent:
\begin{equation}
\Pi(\theta,\theta) = E_P\left[\int_{\underline{\theta}}^\theta \frac{c_1g(e(s))\delta^{\tau(s)}}{g'(e(s))s}ds\right], \label{eq:InfoRent}
\end{equation}
where $\underline{\theta}$ denotes the lowest type that participates, so that $\Pi(\underline{\theta},\underline{\theta})=0$.


\subsection{Revenue-maximizing contract for the owner \label{appx:Opt}}

Continuing to follow \citet{bib:laffonttirole1986} and \citet{bib:board}, we treat the owner's problem as an optimal control problem in which the objective is to find $\tau(\hat{\theta})$ and $e(\hat{\theta})$ such that the expectation of total surplus minus information rent is maximized. We therefore write the owner's problem as
\begin{equation}
\max_{\tau(\theta),e(\theta,P),\underline{\theta}} \int_{\underline{\theta}}^{\bar\theta} \left[E_P\left(P_\tau g(e(\theta))\theta - c_0 - c_1e(\theta)\right)\delta^{\tau(\theta)} - \int_{\underline{\theta}}^\theta \frac{c_1g(e(s))\delta^{\tau(s)}}{g'(e(s))s}ds\right]f(\theta)d\theta, \label{eq:PrincipalProb}
\end{equation}
where the owner also chooses the type $\underline{\theta}$ for which the individual rationality constraint binds with equality.

To eliminate the double integral, we can use Fubini's theorem. Letting $h(\theta)\equiv f(\theta)/(1-F(\theta))$ denote the hazard function, we re-write the owner's problem as:
\begin{equation}
\max_{\tau(\theta),e(\theta,P),\underline{\theta}} \int_{\underline{\theta}}^{\bar\theta} E_P\left[\left(P_\tau g(e(\theta))\theta - c_0 - c_1e(\theta) - \frac{c_1}{\theta h(\theta)} \frac{g(e(\theta))}{g'(e(\theta))}\right)\delta^{\tau(\theta)}\right]f(\theta)d\theta. \label{eq:PrincipalProb2}
\end{equation}

At this point, it is useful to recall the firm's problem, equation (\ref{eq:firmprob}). Following the logic in \citet{bib:board}, the owner can induce the firm to follow the stopping rule implied by (\ref{eq:PrincipalProb2}) by setting the contingent payment $z$ equal to the information rent term in (\ref{eq:PrincipalProb2}), since doing so makes the firm's problem equivalent to the owner's problem. Thus, the revenue-maximizing contingent payment is given by:\footnote{The owner's optimal timing and effort functions as defined by equations (\ref{eq:PrincipalProb2}) and (\ref{eq:eFOC}) will satisfy the firm's second order conditions (the optimal stopping time is decreasing in $\theta$ and $e'(\theta)\geq0$) if the hazard rate is monotonically increasing ($h'(\theta)\geq0$) and if $g'''\leq0$. To see this sufficiency, observe that the total derivative of the term in parentheses in (\ref{eq:PrincipalProb2}) is strictly increasing in $\theta$, via application of the envelope theorem to the firm's problem. Thus, per Lemma 1 in \citet{bib:board}, the optimal stopping time is decreasing in $\theta$. Second, take the derivative of (\ref{eq:eFOC}) with respect to $\theta$ and collect terms involving $e'(\theta)$. It then becomes apparent that $h'(\theta)\geq0$ and $g'''\leq0$ are sufficient for $e'(\theta)>0$.}
\begin{equation}
z_\tau(\theta,y,P^t) = \frac{c_1}{\theta h(\theta)}\frac{g(e(\theta))}{g'(e(\theta))}. \label{eq:ContPayment}
\end{equation}

The optimal contingent payment $z$ from the firm to the owner is positive, which will lead to delayed drilling relative to the social optimum. Note that the optimal payment is zero for the highest type $\bar{\theta}$ firm because $1/h(\bar{\theta})=0$, reflecting the standard ``no distortion at the top'' rule. The optimal up-front payment $R(\theta)$ is set to equate the firm's utility to the information rent expressed in equation (\ref{eq:InfoRent}), where the utility of the endogenously chosen type $\underline{\theta}$ firm is set to zero.

The presence of a contingent payment upon execution of the option echoes \cites{bib:board} result. What differs here is that the optimal payment is contingent not just on drilling but also on effort. Because effort is not observable to the owner, this optimal mechanism must be implemented using a payment that is contingent on production $y$ rather than effort itself. Paralleling \citet{bib:laffonttirole1986}, we now examine an implementation that involves an affine contingent payment: a lump sum transfer combined with a linear tax on production. The appeal of an affine payment is that its optimality is robust to the distribution of the disturbance $\varepsilon$. The downside is that the sufficient conditions for incentive compatibility will be stronger than those discussed in section~\ref{appx:FirmProb} above, since the affine payment structure constrains punishments for deviations outside of the concealment set.\footnote{The sufficient conditions for incentive compatibility of the affine contingent payment (equation (\ref{eq:CPay3})) are difficult to characterize in terms of primitives. In the event that they are not satisfied, the owner will need to ``iron'' over regions in the type space where incentive compatibility does not hold.\label{fn:affinesuf}}

To derive the optimal linear production tax, we first take the pointwise derivative of (\ref{eq:PrincipalProb2}) with respect to $e(\theta)$ to obtain the FOC that defines the owner's optimal effort function, conditional on drilling at $\tau$. Suppressing the dependence of $g(e(\theta))$ and its derivatives on $\theta$, this FOC is given by:
\begin{equation}
FOC_{e(\theta)}: P_\tau g'\theta-c_1-\frac{c_1}{\theta h(\theta)}\frac{g'^2-gg''}{g'^2}=0 \label{eq:eFOC}
\end{equation}

In FOC (\ref{eq:eFOC}), the last term will be positive due to the assumption that $g''<0$, and the first two terms together will equal zero at the socially optimal effort level. Thus, for equation (\ref{eq:eFOC}) to hold, $e(\theta)$ must be strictly less than socially optimal effort except for type $\bar{\theta}$, where the fact that $1/h(\bar{\theta})=0$ implies that the highest type firm will exert optimal effort. Again, this is the standard ``no distortion at the top'' result.

To obtain the optimal production tax, we return to the firm's problem, taking the derivative of equation (\ref{eq:firmprob}) with respect to $e$ to derive the firm's FOC for its optimal effort, conditional on drilling at $\tau$:
\begin{equation}
FOC_e: P_\tau g'\theta-c_1 - \frac{\partial z_\tau(\theta,y,P^t)}{\partial y}g'\theta = 0 \label{eq:FirmeFOC}
\end{equation}

Combining equations (\ref{eq:eFOC}) and (\ref{eq:FirmeFOC}) yields the linear tax on production that aligns the firm's incentives with the effort function that the owner wishes to induce:
\begin{align}
\frac{\partial z_\tau(\theta,y,P^t)}{\partial y} &= \frac{c_1}{\theta^2 h(\theta)g'}\frac{g'^2-gg''}{g'^2} \nonumber \\
&= \frac{c_1}{\theta^2 h(\theta)g'}\left(1-\frac{gg''}{g'^2}\right) \label{eq:ProdTax}
\end{align}

We can now solve for the optimal affine contingent payment using equations (\ref{eq:ContPayment}) and (\ref{eq:ProdTax}):
\begin{equation}
z_\tau(\theta,y,P^t) = \frac{c_1}{\theta h(\theta)}\frac{g}{g'} + \frac{c_1}{\theta^2 h(\theta)g'}\left(1-\frac{gg''}{g'^2}\right) \left( y-g\theta\right) \label{eq:CPay}
\end{equation}

Rearranging, we obtain:
\begin{equation}
z_\tau(\theta,P^t,y) = \frac{c_1g}{\theta h(\theta)g'}\frac{gg''}{g'^2} + \frac{c_1(1-\frac{gg''}{g'^2})}{\theta^2 h(\theta)g'}y \label{eq:CPay2}
\end{equation}

Finally, we may use equation (\ref{eq:FirmeFOC}) to eliminate $c_1$ in equation (\ref{eq:CPay2}) and obtain:
\begin{equation}
z_\tau(\theta,P^t,y) = \frac{gg''P_\tau g\theta}{\theta h(\theta)g'^2+g'^2-gg''} + \frac{(g'^2-gg'')P_\tau y}{\theta h(\theta)g'^2+g'^2-gg''} \label{eq:CPay3}
\end{equation}

Equation (\ref{eq:CPay3}) corresponds to equation (\ref{eq:contpayment}) in section \ref{sec:AnalyticModelDisc} of the main text. The second term in (\ref{eq:CPay3}) is a positive tax on revenue $P_\tau y$; i.e., a royalty. The first term is negative (because $g''<0$) and represents a payment from the owner to the firm at the time of drilling that is not dependent on output. The affine implementation of the optimal mechanism therefore taxes natural gas production and subsidizes drilling, where the tax and subsidy rates are type-dependent. The up-front bonus payment $R(\theta)$ is used to capture the remainder of the available surplus, net of information rents.


\section{Computational model details \label{appx:comp_model_details}}
This appendix describes the model from section~\ref{sec:CompModel} in greater detail.

The model has five state variables. The first is the firm's type, $\theta$, which is known to the firm but not to the mineral owner. The second is the stage $\ell$ the lease is in and the associated contract characteristics $\chi_{\ell}$. The third is the period, $t$, which directly affects the firm's continuation payoffs during the primary term. The fourth is the time-specific natural gas price $P_t$, and the fifth is the expected drilling and completion cost $C_t$, both of which evolve stochastically. We use the subscript triplet $\{\ell,t,\theta\}$ to denote the three deterministic states, and we explicitly list the stochastically-evolving states. That is, $V^a_{\ell,t,\theta}(P_t,C_t)$ denotes the expected value to the firm of taking action $a$ given lease regime $\chi_{\ell}$, time $t$, firm type $\theta$, and the current gas price and drilling cost $\{P_t,C_t\}$.

\subsection{Well payout and the static profit function}

The static drilling profits accruing to the firm differ depending on whether its profits before taxes, royalties, and operating costs are positive (i.e., whether the well ``pays out''). As discussed in section \ref{sec:Inst}, unleased mineral interests are not liable for well costs if the well fails to pay out. In addition, severance taxes are waived. These rules create a kink in the profit function at the payout point. This kink also affects the optimal amount of water use, since the firm's first-order condition that determines optimal water use will depend on whether the firm expects the well to pay out or not.

If a well drilled at time $t$ ``pays out'', its profits are given by equation (\ref{eq:static_pi}):
\begin{equation}
\pi_t(W) = (1-\tau)(\gamma(1-s)(1-k)P_t-c)\theta W^{\beta} - \gamma(1-s+sk)((1-\tau) P_w W + (1-\tau_c) C_t) \label{eq:static_pi}
\end{equation}

\noindent where $\gamma$ denotes the share of acreage leased, $s$ denotes the severance tax, $k$ denotes the unit's royalty,\footnote{Within each unit we assume there is a single, time-invariant unit royalty equal to the acreage-weighted average royalty across the unit's leases.} $c$ denotes operating expenses, $C_t$ denotes drilling and completion costs other than water use, $\tau$ is the income tax rate, and $\tau_c$ is the effective income tax rate on capital expenditure.

If the well does not pay out, the firm does not pay severance taxes or make payments to unleased mineral interests, so that profits are instead given by equation (\ref{eq:static_pi_nopay}):
\begin{equation}
\pi_t(W) = (1-\tau)((1-\gamma k)P_t-c)\theta W^{\beta} - ((1-\tau) P_w W + (1-\tau_c) C_t) \label{eq:static_pi_nopay}
\end{equation}

Given the parameter inputs to the static profit function, we determine the optimal water use $W^*$ by first finding the values $W_{+}$ and $W_{-}$ that maximize equations (\ref{eq:static_pi}) and (\ref{eq:static_pi_nopay}), respectively. If $W_{+}$ results in positive payout, we set $W^*=W_{+}$. Alternatively, if $W_{-}$ results in negative payout, we set $W^*=W_{-}$. Finally, it is possible that $W_{-}$ results in positive payout while $W_{+}$  results in negative payout. In that case we interpolate the value of $W^*\in(W_{+},W_{-})$ that results in zero payout.

Given $W^*$, we then compute the firm's static drilling profits using either equation (\ref{eq:static_pi}) or equation (\ref{eq:static_pi_nopay}), depending on the payout condition.

\subsection{Solving the dynamic model}
The model is solved with backward induction. We therefore discuss the solution in reverse order.

\subsubsection{Extension phase ($t \geq \bar{T} + 1$, $\ell=2$): the firm's problem}
\label{subsub:firm_extension}

In the extension phase ($\ell=2$), the firm can drill ($a=1$) or not drill ($a=0$). Conditional on the price and cost states, the value function is stationary with respect to $t$. Hence, we do not index the extension period's value function by $t$.

The expected payoff from drilling in period $t$ and in lease regime $\ell=2$, with firm type $\theta$, is determined by contract terms $k_{2}$ and $S_2$, the share of land leased $\gamma_{t}$, the firm's expected present value of production $\theta$, the natural gas price $P_t$, and the drilling and extraction cost $C_t$ is:

\begin{equation}
V^1_{\ell=2,\theta}(P_t,C_t) = \pi^*_{\ell=2, \theta}(P_t,C_t) + S_{2}
\label{eq:one_period_profits_drilling}
\end{equation}

If the firm does not drill ($a=0$), it continues in the game and receives the continuation value:

\begin{equation}
V^0_{\ell=2,\theta}(P_t,C_t) = \delta E [V_{\ell=2,\theta}(P_{t+1},C_{t+1})|P_t,C_t]
\end{equation}

The continuation value $E [ V_{\ell=2,\theta}(P_{t+1},C_{t+1})|P_t,C_t]$ is the expected value of next period conditional on this period's observable states, and the expectation is taken over all possible draws of shocks and the transition of output price and costs. 

The cost shocks each period are denoted $\nu^a_t$ for each action $a$, and they additively affect the payoff of each action. The firm's total payoff for choosing action $a$ in period $t$ is then $V^a_{\ell=2,\theta}(P_t,C_t) + \nu^a_t$. The expected value to the firm at the ``start'' of a given period $t$, before the $\nu^a_t$ shocks are realized and the action is chosen, is then:

\begin{equation}
E[V_{\ell=2,\theta} (P_t,C_t)] = E [ \max\{ V^1_{\ell=2,\theta}(P_t,C_t) + \nu^1_t, \; V^0_{\ell=2,\theta}(P_t,C_t) + \nu^0_t \}]
\label{eq:expected_unconditional_value_i_eq_2}
\end{equation}

We assume that $\nu^{1}_t$ and $\nu^{0}_t$ are i.i.d. draws from a type 1 extreme value distribution with mean zero and scale parameter $\sigma_{\nu}$. With these distributional assumptions on the shocks $\nu^a_t$, the probability of drilling in period $t$ conditional on not having drilled before is:

\begin{equation}
p^d_{\ell=2,\theta}(P_t,C_t)  =
\dfrac{ \exp \left( \dfrac{V^1_{\ell=2, \theta}(P_t,C_t)}{\sigma_{\nu}} \right) }{ \exp \left( \dfrac{V^1_{\ell=2,\theta}(P_t,C_t)}{\sigma_{\nu}} \right) + \exp \left( \dfrac{V^0_{\ell=2,\theta}(P_t,C_t)}{\sigma_{\nu}} \right)  }
\label{eq:extensiondrillingprobs}
\end{equation}

We solve for the value functions and drilling probabilities using value function iteration.


\subsubsection{End of Primary Term ($t = \bar{T}, \ell=1$): the mineral owner's problem:} \label{subsub:owner_endpriterm}

When deciding which contract terms to offer for the extension, the mineral owner predicts how the firm will respond to its choice of contract terms $\chi_2$. For every possible price path, for each period $t \geq \bar{T} + 1$, and for each type $\theta$, the mineral owner forecasts the probability of each of the firm's actions $a$ and the effect of each action on the mineral owner's payoff. Specifically, if firm type $\theta$ drills ($a = 1$) in period $t$ in lease regime $\ell=2$, the mineral owner receives a royalty payment and pays the drilling subsidy per the contract terms $\chi_2$. If the firm continues in the lease, the mineral owner receives zero in that period. Using the drilling probabilities from equation (\ref{eq:extensiondrillingprobs}) and the state transition matrix for $P_t$ and $C_t$, the owner can calculate its expected present value payment for each firm type $\theta$, given contract terms $\chi_2$.

As discussed in the main text, we assume that the royalty $k_2$ and subsidy $S_2$ are the same as those from the primary term, consistent with practice in the Haynesville. The owner's problem is then simply to choose the optimal extension bonus $R_2$. Because the owner can calculate both its own expected value from the continuation period as well as the expected values $V^0_{\ell=1,\bar{T},\theta}$ for each type $\theta$, this problem is equivalent to an optimal reserve price problem. As noted in the main text, we make the tractability assumption that the distribution of types that the owner believes it faces at $\bar{T}$ is the same as the initial distribution at $t=0$.



\subsubsection{End of primary term ($t = \bar{T}$, $\ell=1$): the firm's problem} \label{subsub:firm_endpriterm}

In $t=\bar{T}$, the firm's problem is similar to the case of $t \geq \bar{T} + 1$, with three exceptions. First, the lease is in the last phase of the primary term in which $\ell=1$, meaning that if the firm drills, its payoffs depend on $\chi_1$ rather than $\chi_2$. Second, the value function is non-stationary during the primary term. That is, because the firm can extend or abandon the lease at time $\bar{T}$, the value function is a function of $t$ itself. Third, rather than choosing between drilling and waiting, the firm instead chooses between drilling, abandoning the lease, and signing the extension. We assume that the random shock for abandoning is the same as the random shock for extending and is equal to $\nu^0_{\ell=1,\bar{T}}$. Therefore we can treat the firm's problem as a nested problem in which the firm first chooses whether to drill or not drill. Then conditional on not drilling, the firm must decide whether to abandon or to extend. As above, the action to not drill is denoted as $a=0$, and the value of not drilling is:
\begin{equation}
V^0_{\ell=1,\bar{T},\theta}(P_{\bar{T}},C_{\bar{T}}) = \max \{ -R_2 + \delta E [V_{\ell=2,\theta}(P_{\bar{T}+1},C_{\bar{T}+1})|P_{\bar{T}},C_{\bar{T}}], 0 \}
\end{equation}

The firm's expected value at the ``start'' of period $t=\bar{T}$, before the $\nu^a_{\bar{T}}$ shocks are realized and the action is chosen, is then given by the expected value of the maximum over the choices to drill and not drill:
\begin{equation}
E [V_{\ell=1,\bar{T},\theta}(P_{\bar{T}},C_{\bar{T}})] =
E [ \max\{ V^1_{\ell=1,\bar{T},\theta}(P_{\bar{T}},C_{\bar{T}}) + \nu^1_{\bar{T}}, \; V^0_{\ell=1,\bar{T},\theta}(P_{\bar{T}},C_{\bar{T}}) + \nu^0_{\bar{T}} \}]
\label{eq:expected_unconditional_value_t_eq_Tbar}
\end{equation}

With our distributional assumptions on the shocks $\nu^a_t$, the probability of drilling in period $t = \bar{T}$ conditional on not having drilled before is:
\begin{equation}
p^d_{\ell=1,\bar{T},\theta}(P_{\bar{T}},C_{\bar{T}}) =
\dfrac{ \exp \left( \dfrac{V^1_{\ell=1,\bar{T},\theta}(P_{\bar{T}},C_{\bar{T}})}{\sigma_{\nu}} \right) }{ \exp \left( \dfrac{V^1_{\ell=1,\bar{T},\theta}(P_{\bar{T}},C_{\bar{T}})}{\sigma_{\nu}} \right) + \exp \left( \dfrac{V^0_{\ell=1,\bar{T},\theta}(P_{\bar{T}},C_{\bar{T}})}{\sigma_{\nu}} \right)  }
\end{equation}

Similarly, the probability of extending, given that the firm has not drilled previously, can be written as below, where the indicator term indicates whether the firm would prefer to extend rather than abandon:
\begin{equation}
p^e_{\ell=1,\bar{T},\theta}(P_{\bar{T}},C_{\bar{T}}) =
\dfrac{ \exp \left( \dfrac{V^0_{\ell=1,\bar{T},\theta}(P_{\bar{T}},C_{\bar{T}})}{\sigma_{\nu}} \right) }{ \exp \left( \dfrac{V^1_{\ell=1,\bar{T},\theta}(P_{\bar{T}},C_{\bar{T}})}{\sigma_{\nu}} \right) + \exp \left( \dfrac{V^0_{\ell=1,\bar{T},\theta}(P_{\bar{T}},C_{\bar{T}})}{\sigma_{\nu}} \right)  }  \cdot 1\bigg[ V^0_{\ell=1,\bar{T},\theta}(P_{\bar{T}},C_{\bar{T}}) \geq 0 \bigg]
\end{equation}




\subsubsection{Primary term $1 \leq t \leq \bar{T} - 1$: firm's problem:}

During the periods in which $1 \leq t \leq \bar{T} - 1$, the firm can choose between drilling ($a=1$) and waiting ($a=0$). We can write the same value functions and choice probabilities as in section~\ref{subsub:firm_extension}, but with two differences. First, the lease regime is $\ell=1$ rather than $\ell=2$, meaning the payoffs from drilling depend on contract terms $\chi_1$.  Second, as mentioned in section~\ref{subsub:firm_endpriterm}, the finite primary term means that the value function is a function of $t$. The payoff from waiting and therefore the overall payoff each period also depend on the firm's expectations of how the mineral owner will choose $\chi_2$ (and in particular $R_2$).

We solve for the firm's drilling probabilities and value function in each period of the primary term by iterating backwards from the values calculated in section~\ref{subsub:firm_endpriterm}.


\subsubsection{Initial time period $t = 0$: firm's problem}

Given the initial contract terms $\chi_1$ that the mineral owner offers the firm, the firm will choose to accept and pay the initial bonus $R_1$ if the expected profits exceed the bonus payment. Therefore the firm accepts if:

\begin{equation}
\delta E [V_{\ell=1,1,\theta}(P_1,C_1)|P_0,C_0] \geq R_1
\end{equation}

We assume that there are no shocks here, which implies that there will be a threshold value $\theta^*(\chi_1)$ such that $ \delta E [V_{\ell=1,1,\theta^*(\chi_1)}(P_1,C_1)|P_0,C_0] = R_1$, and all firm types $\theta \geq \theta^*(\chi_1)$ will accept the lease.

\subsubsection{Initial time period $t = 0$: mineral owner's problem}

In the initial period, the mineral owner observes the current market price and cost conditions, and chooses contract terms $\chi_1$, including the primary term $\bar{T}$, to maximize the expected sum of payoffs during the primary term, from any extension payments, and payoffs during the extension. In our counterfactuals, we sometimes fix the royalty, drilling subsidy, or primary term components of $\chi_1$, but we always model the owner as choosing the initial bonus $R_1$ to maximize its payoff (given the other components in $\chi_1$).


\subsection{Additional wells \label{appx:additional_wells}}

In some analyses we augment the model so that the lease can accommodate multiple wells. Consistent with practice in the Haynesville, we assume the firm needs to only drill one well to hold the lease. Once one well is drilled, the lease is held by production, and additional drilling can happen at any later time period. We assume that the total number of wells that can be drilled on the lease is $M$.

To simplify the analysis, we make some assumptions on the functional form of payoffs. We assume that there is an additive shock $\nu^1_t$ that applies to each well that the firm drills in period $t$ and another additive shock $\nu^0_t$ that applies to each (remaining) possible well that the firm could drill but does not drill in period $t$. We also assume no economies of scale in drilling costs and that any cost subsidy $S_{\ell}$ is paid for each well. Therefore the total one-time payoff for drilling $N$ wells out of a total of $M$ possible wells is is $N [\pi^*_{\ell,\theta}(P_t,C_t) + S_{\ell} + \nu^1_t] + (M - N) \nu^0_t$.

These additivity assumptions allow us to significantly decrease the number of choices we need to consider. Flow profits increase linearly in the number of wells drilled. Thus, if the lease has no expiration date, whether because it is held by production or due to a lease extension, the firm will always drill either zero wells or all possible wells. 

However, if the lease is in its primary term and is not held by production, we must also consider the possibility that the firm would prefer to drill a single well. Drilling a single well extends the lease indefinitely; such an extension would otherwise require an additional bonus payment. Thus, the incremental payoff from the first well is larger than that of subsequent wells, and total payoff is no longer linear in the number of wells drilled.  However, beyond that first well, the incremental payoffs are again constant, so the firm would never choose to drill strictly between one and $M$ wells.

In this extension of the model, we need to define some additional notation. We use the tilde to denote the firm's per-well continuation value, equal to the total continuation value divided by the number of total remaining wells that can be drilled on the lease.
\begin{itemize} 
	\item Let $E[\tilde{V}_{\ell,t+1,\theta}|P_t,C_t])$ denote the firm's per-value continuation value at $t$ if the lease has not been held by production at period $t$.
	\item Let $E[\tilde{W}_{\ell,t+1,\theta}|P_t,C_t])$ denote the firm's per-well continuation value if the lease has been held by production by date t (i.e., if there has been drilling at $t$ or before).\footnote{Once the well is held by production, the problem becomes stationary: $\tilde{W}$ and $\pi^*$ only vary over time through the arguments $P_t$ and $C_t$. However, in this subsection, we use the $t$ subscript for $\tilde{V}$, $\tilde{W}$, and $\pi^*$, suppressing the arguments $P_t$ and $C_t$ to save space.} 
\end{itemize}
Because the constraint of a primary term decreases the expected continuation value of the lease, $E[\tilde{V}_{\ell,t+1,\theta}|P_t,C_t] < E[\tilde{W}_{\ell,{t+1},\theta}|P_t,C_t]$.

We focus here on the primary term: periods $t \in [1, \bar{T} - 1 ]$ for which no drilling has yet occurred. Once the lease is extended indefinitely by drilling or making an extension payment, the problem becomes nearly identical to the single well case, so we omit these cases from the exposition. As discussed above, the firm has three choices. First, the firm may choose to drill all $M$ wells ($a = M$), thus ending the optimal stopping problem. The firm receives $M$ times the per-well profits.

\begin{equation}
V^M_{\ell,t,\theta}(P_t,C_t) = M [\pi^*_{\ell,t,\theta} + S_{\ell} ] \label{eq:multipayoffM}
\end{equation}

Second, the firm may choose to drill only one well ($a = 1$), thus holding the lease. As in the single-well case, the firm receives the profits from that decision. In addition, the firm receives the continuation payoff associated with the option to drill $M - 1 $ future wells:

\begin{equation}
V^1_{\ell,t,\theta}(P_t,C_t) = \pi^*_{\ell,t,\theta} + S_{\ell} + \delta  (M - 1) E[\tilde{W}_{\ell,t+1,\theta}|P_t,C_t]
\label{eq:multipayoff1}
\end{equation}

Finally, the firm can retain the lease without drilling ($a = 0$). It gets the continuation value associated with an undrilled lease capable of holding $M$ wells.

\begin{equation}
V^0_{\ell,t,\theta}(P_t,C_t) = \delta M E[\tilde{V}_{\ell,t+1,\theta}(P_{t+1},C_{t+1})|P_t,C_t] \label{eq:multipayoff3}
\end{equation}

And, as discussed above, we also assume additive shocks to the firm's payoff: for each well that the firm drills, it gets the shock $\nu^1_t$.  For each potential well that it does not drill, it gets $\nu^0_t$. Thus the full payoffs from each of the three choices are:
\begin{equation}
\begin{aligned}
\mbox{Drill all $M$ wells:                 } & V^M_{\ell,t,\theta}(P_t,C_t)  + M \nu^1_t \\
\mbox{Drill 1 well:                 }                               & V^1_{\ell,t,\theta}(P_t,C_t)  + \nu^1_t + (M - 1) \nu^0_t \\
\mbox{Drill zero wells (continue):      } & V^0_{\ell,t,\theta}(P_t,C_t)  + M  \nu^0_t
\end{aligned}\label{eq:multiwell-shocks}
\end{equation}

This structure leads to a tractable set of choice probabilities. Let $u^1_{\ell,t,\theta} = \pi^*_{\ell,t,\theta} + S_{\ell}$ denote the deterministic payoff from drilling a single well.  Combining equation (\ref{eq:multiwell-shocks}) with equations (\ref{eq:multipayoffM})--(\ref{eq:multipayoff3}), we can characterize the firm's choice probabilities. The firm prefers drilling all $M$ wells to a single well if:
\begin{equation}
\nu^1_t - \nu^0_t > \delta E[\tilde{W}_{\ell,t+1,\theta}|P_t,C_t] - u^1_{\ell,t,\theta}.
\end{equation}

\noindent In addition, the firm prefers drilling  $M$ wells to not drilling any wells if:
\begin{equation}
\nu^1_t - \nu^0_t > \delta E[\tilde{V}_{\ell,t+1,\theta}|P_t,C_t] - u^1_{\ell,t,\theta}.
\end{equation}
\noindent Since $E[\tilde{W}_{\ell,t+1,\theta}|P_t,C_t] > E[\tilde{V}_{\ell,t+1,\theta}|P_t,C_t]$, it is clear that if the firm prefers drilling $M$ wells to drilling one well, it also prefers drilling $M$ wells to drilling zero wells.

Finally, the firm prefers drilling one well to drilling no wells if:
\begin{equation}
\nu^1_t - \nu^0_t > M \delta E[\tilde{V}_{\ell,t+1,\theta}|P_t,C_t] - (M - 1) \delta E[\tilde{W}_{\ell,t+1,\theta}|P_t,C_t] - u^1_{\ell,t,\theta} \\
\end{equation}

This system of preferences is then an ordered logit. There are three distinct intervals on the support of $\nu^1_t - \nu^0_t$, each associated with a single optimal action: drill all $M$ wells, drill 1 well, or drill no wells. This result produces some intuitive firm behavior. Suppose that the lease is early in the primary term.  Then $E[\tilde{W}_{\ell,t+1,\theta}|P_t,C_t]$ and $E[\tilde{V}_{\ell,t+1,\theta}|P_t,C_t]$ are reasonably close to one another because the value of moving into the indefinite secondary term immediately is low (if they were indeed equal, the interval on which $a=1$ is chosen would be zero). Thus, conditional on wanting to drill at least one well early on in the lease, the firm is likely to drill all of them. On the other hand, suppose $E[\tilde{W}_{\ell,t+1,\theta}|P_t,C_t]$ is much larger than $E[\tilde{V}_{\ell,t+1,\theta}|P_t,C_t]$ because the lease is late in the primary term. Then the payoff from moving into the secondary term is large, and the firm will be more likely to drill just one well conditional on drilling any at all.

The expression is somewhat more complicated when we move to the case where the primary term is about to expire ($t = \bar{T}$). Now the firm's choice to drill zero wells now includes two possible sub-actions: pay to extend, or abandon. Now we write the value of drilling zero wells as:

\begin{equation}
V^0_{\ell,\bar{T},\theta}(P_{\bar{T}},C_{\bar{T}}) = \max \{ - R_2 + \delta M E[\tilde{V}_{\ell,\bar{T}+1,\theta}|P_{\bar{T}},C_{\bar{T}}] , 0 \}
\end{equation}

Again, the firm's decision probabilities can be solved as an ordered logit, in which the firm chooses between drilling all $M$ wells, drilling one well, or drilling zero wells. Then, conditional on drilling zero wells, the firm continues only if $V^0_{\ell,\bar{T},\theta}(P_{\bar{T}},C_{\bar{T}}) \geq 0$.


\section{Additional details on model calibration \label{appx:calibration}}

This appendix provides the details of our model calibration procedures, summarized in section \ref{sec:Calibration} in the main text.

\subsection{Production function estimation}

Our production function depends on a geographically varying productivity parameter $\theta(lon_i,lat_i)$ and the coefficient on water use, $\beta$. We use a partially linear model to smoothly predict productivity at the centroid of each potential unit. We do this for two reasons. First, some units did not have drilling, and this method allows us to predict unobserved productivity on those units. The second is that our measure of expected production $Y_i$ is noisy, and our smoothing approach gives a more reasonable measure of a firm's expected productivity.

We implement the difference estimator suggested in \citet{bib:robinson}. We model production as:

\begin{equation}
\ln(Y_i) = \Theta(lon_i,lat_i) + \beta w_i + u_i
\end{equation}

\noindent where $w_i = \ln(W_i)$ is the log of the amount of water used in fracking well $i$, and $\Theta(\cdot)$ is a nonparametric function of well coordinates $\{lon_i,lat_i\}$.

% Note: exact implementation comes from the Cameron and Trivedi textbook (2005)

First, we estimate a nonparametric regression of $\ln(Y_i)$ on the location of each well to obtain predicted values ($m^y_i$) and residuals ($e^y_i$).\footnote{
	In each nonparametric regression that follows, we use a Gaussian kernel of functional form $\phi(0, \sigma^2_{\phi,x})$, where $\phi$ is the normal pdf with a mean of $0$ and a standard deviation of $\sigma_{\phi,x}$. For each variable of interest $x$, we compute a separate optimal bandwidth using a leave-one-out estimator. We use a weighted average of all wells $i^{\prime}$ that are not in $i$'s unit to predict the value of $x$ for well $i$. The optimal bandwidth is the value that minimizes the sum of the squares of the differences between the actual and predicted values of $x_i$:
	
	\begin{equation}
	\sigma_{\phi,x} = \argmin_{\sigma > 0} \sum_i  \left( x_i -\dfrac{ \sum_{i^{\prime} \notin u(i)} \phi( d(i, i^{\prime}), \sigma^2) \cdot x_{i^{\prime}}  }{ \sum_{i^{\prime} \notin u(i)} \phi( d(i,i^{\prime}), \sigma^2) } \right)^2
	\end{equation}
	
	\noindent In the final regressions (but not during cross-validation), we also apply a caliper that assigns zero weight to any observations greater than four bandwidths away from the observation of interest.

} Second, we estimate a nonparametric regression of $w_i$ on the location of each well to obtain predictions ($m^{w}_i$) and residuals ($e^{w}_i$). Third, we regress $e^y_i$ on $e^{w}_i$.  This last regression returns an estimate of the production function coefficient $\hat{\beta}$ that accounts for the fact that different amounts of water were used in different areas on average. Fourth, we subtract the effect of water from each well's smoothed predicted productivity:
\begin{equation}
\hat{m}_i = m^y_i - \hat{\beta} m^{w}_i
\end{equation}

\noindent We nonparametrically smooth these values to predict the productivity of a well drilled at the centroid of unit $u$.

\begin{equation}
\ln(\theta_{lon_u,lat_u}) = \widehat{\Theta}(lat_u,lon_u) = \dfrac{ \sum_{i^{\prime} \notin u} \phi( d(u, i^{\prime}), \sigma^2_{\phi,u}) \cdot \hat{m}_i  }{ \sum_{i^{\prime} \notin u} \phi( d(u, i^{\prime}), \sigma^2_{\phi,u}) }
\end{equation}

\noindent Here $u$ indexes units and $d(i^{\prime}, u)$ is a measure of distance between a well $i$ and the centroid of a unit $u$.

\subsection{Restrictions imposed in the calibration sample}

As noted in section \ref{sec:Calibration}, we impose restrictions on the sample of units used to calibrate the water price $P_w$, the level of drilling costs and their sensitivity to the rig dayrate $D_t$, and the scale parameter $\sigma_\nu$ governing the variance of the idiosyncratic cost shocks $\nu^a_t$. These restrictions are designed to reduce the influence of measurement error in unit-level productivity, royalties, and leased acreage.

Starting with the sample of      891units that were actively leased by 1Q 2010 but in which drilling had not yet occurred by 1Q 2010, we first filter out units for which there are fewer than two nearby wells in the productivity estimation caliper and units with no royalty data. These restrictions affect       91and       48units, respectively. 

We then drop units in which reported leased acreage ever increases after 1Q 2010, affecting      518out of the original      891units. This restriction is necessary because the process of adding acreage (and bargaining with multiple landowners) is outside the scope of our model. We then mitigate problems with measurement error in reported acreage by dropping units never having more than 160 acres leased in our data (affecting      258out of the original      891units) and then re-scaling each unit's leased acreage in each quarter so that the maximum acreage leased in each unit during the sample period is 640 acres (the standard unit size). That is, we multiply the leased acreage in each unit $i$ in quarter $t$ by 640 and divide by the maximum reported acreage for $i$ during the sample. This re-scaling is motivated by our belief that reported \emph{changes} in relative acreage leased within a unit over time, particularly when changes are driven by lease expiration, are less error-prone that reported \emph{levels} of acreage.

Finally, we drop units in which a well is drilled when leased acreage is zero in our data. This last restriction removes        4units, following the imposition of the other restrictions. It is necessary because our model can never rationalize drilling when acreage is zero (since the firm makes no profits), but we sometimes see zero acreage in the data due to missing and mis-reported data in the lease records. The final calibration sample then contains      160units. Figure \ref{fig:calibration_unit_map} maps the distribution of these units within the Haynesville Shale.

% FIGURE SHOWING MAP WITH CALIBRATION SAMPLE
\begin{figure}[!t]
\centering
\figcapt[\textwidth]{Locations of units in the calibration sample}
\includegraphics[scale=0.5]{final_estimation_sample_unit_map.pdf}
\label{fig:calibration_unit_map}
\end{figure}



\subsection{Static profit function}

The static profit function that we use throughout our calibration and counterfactual simulations accounts for unit-level royalties and taxes, leased acreage, and productivity, along with the gas price and rig dayrate prevailing at the time of drilling. If the well ``pays out'' (where the pay out condition is based on profits before any taxes, royalties, or operating costs),\footnote{When testing the payout condition, we compute non-water drilling costs $C_t$ using the rig dayrate applicable to each well and estimates of $\alpha_0$ and $\alpha_1$ obtained from estimating the projection in equation (\ref{eq:cost_day_rate_relationship}), given $P_w$.} profits are given by equation (\ref{eq:static_pi}) in appendix \ref{appx:comp_model_details}, which we reproduce here:
\begin{equation}
\pi_t(W) = (1-\tau)(\gamma(1-s)(1-k)P_t-c)\theta W^{\beta} - \gamma(1-s+sk)((1-\tau) P_w W + (1-\tau_c) C_t) 
\end{equation}

\noindent where $\gamma$ denotes the share of acreage leased, $s$ denotes the severance tax, $k$ denotes the unit's royalty,\footnote{Within each unit we assume there is a single, time-invariant unit royalty equal to the acreage-weighted average royalty across the unit's leases.} $c$ denotes operating expenses, $C_t$ denotes drilling and completion costs other than water use, $\tau$ is the income tax rate, and $\tau_c$ is the effective tax rate on capital expenditure.

If the well does not pay out, the firm does not pay severance taxes or make payments to unleased mineral interests, so that profits are instead given by equation (\ref{eq:static_pi_nopay}), which we reproduce here:
\begin{equation}
\pi(W) = (1-\tau)((1-\gamma k)P_t-c)\theta W^{\beta} - ((1-\tau) P_w W + (1-\tau_c) C_t) 
\end{equation}



\subsection{Log-likelihood computation for estimating $\alpha_0$ and $\sigma_\nu$}

To compute the probability $pr_{it}$ that a given unit $i$ is drilled in a particular quarter $t$ (or not drilled at all), we use our model to run unit-specific simulations that incorporate our unit-specific productivities and royalties, the actual time path of gas prices and dayrates (which are common across units), and the time path by which each unit's leased acreage is expected to expire. For each lease, we model the expiration date as lease's explicit primary term plus any built-in extension period. We assume the extension payment (which is not observed in our data) is zero to simplify the computation and because we do not observe bunching of drilling prior to the start of the extension period in our data (recall figure \ref{fig:extension_vs_not}). Reductions in leased acreage reduce the firm's profits from drilling, per equation (\ref{eq:static_pi}), which generates an incentive in our model for firms to drill prior to expiration in order to retain acreage for subsequent drilling. To maintain the tractability of these simulations, in which leased acreage can vary continuously between 0 and 640, we shut down the possibility of a lease extension offer upon expiration of acreage. Thus, we force acreage to expire if it is not drilled in time. This assumption substantially increases tractability (we avoid modeling the rich state space of all combinations of extended vs. expired acreage) but may exaggerate the incentive to drill prior to lease expiration. 

We assume that firms anticipate being able to drill up to        3wells per unit, equal to the average number of wells in each unit of our sample in which at least 2 wells were drilled. In our simulations, once a firm drills the first well in its unit, its leased acreage at that time is maintained indefinitely, and this acreage is used to simulate the value of drilling the additional wells in future periods.

The log-likelihood objective function is given by equation (\ref{eq:LL}), where $d_{it}$ is a 0/1 flag for whether unit $i$ was drilled in quarter $t$, and $pr_{i0}$ is the simulated probability that the unit is never drilled.

\begin{equation}
LL(\alpha_0, \sigma_{\nu}; \text{data}) = \sum_{i} \left({ \sum_{t} {(\ln (pr_{it}(\alpha_0, \sigma_{\nu}; \text{data})) \cdot d_{it})}} + \ln(pr_{i0})\cdot (1-\sum_{t}{d_{it}}) \right) \label{eq:LL}
\end{equation}

\subsection{Sensitivity of model output to alternative input parameters}

Table \ref{tab:sensitivity} shows how the mineral owner's optimal royalty and primary term vary with parameter inputs to the model. Row 1 reproduces the main results  from our baseline calibration. In rows 2 and 3, we reduce the owner's uncertainty over the firm's type by reducing the standard deviation of log productivity $\sigma_\theta$ by 50\% and 75\%, respectively. Row 4 reduces the production coefficient $\beta$ on water input by 50\%. Rows 5 and 6 increase and decrease, respectively, the expected productivity of each unit by 33\%. Row 7 reduces the scale $\sigma_{\nu}$ of the logit shocks by 50\%. And row 8 reduces the drilling cost intercept $\alpha_0$ to the value obtained from projecting reported costs, less water costs, onto the rig dayrate per equation (\ref{eq:cost_day_rate_relationship}).

% SENSITIVITY TABLE
\begin{table}[htbp]
\centering
\caption{Sensitivity of optimal royalty and primary term to input parameters}
\begin{tabular} {c l c c c } \midrule \midrule 
 & & & & Increase in \\ 
 & & Optimal & Optimal &  owner's value vs. \\ 
Row & Parameters & royalty & pri term & royalty-only lease \\ 
\midrule 
1 & Baseline calibration from table \ref{tab:sum_calibration} &       53\% &     4.75 years &      2.4\% \\ 
2 & Reduce productivity std dev $\sigma_\theta$ by 50\%$^1$ &       46\% &     5.50 years &      2.4\% \\ 
3 & Reduce productivity std dev $\sigma_\theta$ by 75\%$^1$ &       38\% &     6.25 years &      1.9\% \\ 
4 & Reduce water coefficient $\beta$ by 50\%$^2$ &       60\% &     5.00 years &      3.4\% \\ 
5 & Increase expected productivity by 33\%$^3$ &       52\% &     4.00 years &      2.7\% \\ 
6 & Reduce expected productivity by 33\%$^3$ &       57\% &     6.25 years &      1.9\% \\ 
7 & Reduce scale $\sigma_\nu$ of iid shocks by 50\%$^4$ &       42\% &     3.25 years &      1.6\% \\ 
8 & Use projection (\ref{eq:cost_day_rate_relationship}) to estimate $\alpha_0$$^5$ &       52\% &     2.25 years &      3.0\% \\ 
\midrule 
\end{tabular}
\tabnote[\textwidth]{Notes:\\$^1$When we reduce $\sigma_\theta$, we also increase $\mu_\theta$ so that expected productivity E[$\theta$] in levels is held constant.\\
$^2$When we reduce $\beta$, we increase $\mu_\theta$ to hold production at the average water use constant, and we re-estimate the water price $P_w$.\\
$^3$We shift $\mu_\theta$ so that average productivity increases or decreases by 33\%.\\
$^4$When we reduce $\sigma_\nu$, we re-estimate $\alpha_0$ using maximum likelihood.\\
$^5$When we fix $\alpha_0$ using the drilling cost projection (\ref{eq:cost_day_rate_relationship}), yielding an estimate of $\alpha_0$ of $\$      7.2 $million, we re-estimate $\sigma_{\nu}$ using maximum likelihood.}

\label{tab:sensitivity}
\end{table}

\end{document}
